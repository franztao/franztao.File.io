\section{The complete survivable embedded virtual network embedding algorithm}
In this section, we first present the procedure of augmented resource allocation  of SeVN graph.
%, then we also give out resource demand allocation method fitting for both FD-SeVN and FI-SeVN embedding, since the difference between FD-SeVN and FI-SeVN in embedding approach could be reconciled by a general resources sharing constraint.
More details would be elaborated in the following part.

Besides, with respect to the node embedding even and virtual link embedding, because not all the virtual links or not all their bandwidth would be employed simultaneously under single one node failure, some virtual links could share substrate resources if they are embedded on the same substrate link, which would reduce the total substrate bandwidth needed. In short, backup bandwidth of substrate link corresponding different virtual network request could be shared each other.


\subsection{Embed Augmented Backup Resource into Substrate network}
From Sec.\ref{lab:DynamicProgrammingEquation}, we have obtained node mapping and edge mapping relationship with respect to node $v_i$ failure, and compute out how much computation of every node should to be reallocated  and how much bandwidth of every link should to be reallocate  in substrate network SN as shown in Fig.\ref{fig:AugmentResource}. Augmenting computation resource 0,2,0,0,3,6,0 in physical node $s_1,s_2,s_3,s_4,s_5,s_6,s_7$, respectively. Finding five path with bandwidth 1,4,6,3,6 corresponding to physical path $p_{s_1s_2},p_{s_1s_3},p_{s_1s_5},p_{s_1s_6},p_{s_2s_5}$, respectively. Even though there were most virtual network embedding algorithms, node mapping phase had been completed with respect to our algorithm, the next procedure described edge mapping phase. We use standard shortest path algorithm-dijkstra algorithm\cite{skiena1990dijkstra} to request augmented path of substrate node , and reallocate some bandwidth to these paths of substrate network for this virtual network survivable request. some research\cite{yu2008rethinking} focus on path embedding and path split embedding in order to achieve high link utilization, link stress and other objective target, but we do not focus the edge mapping problem in the paper.

\begin{figure}
  \centering
  % Requires \usepackage{graphicx}
  \includegraphics[width=2in]{Fig/AugmentResource}\\
  \caption{Augmented Resource}\label{fig:AugmentResource}
\end{figure}

Finally, we iteratively repeat the procedure of graph decomposition and multiple knapsack problem for  following each physical node embedded virtual node failure to re-construct the augment graph for guaranteeing virtual network survivable request. As shown in Fig.\ref{fig:Node4Failure}, partial augmented resource which is labeled consecutively with different color of the SeVN illustrate each node failure is presented consecutively. It is worth noting that, every step of partial augmented resource is based on the previous step of augmenting resource.

\subsection{Algorithm procedure}
In this section, we describe our complete algorithm procedure of SeVN problem as shown in Alg.\ref{alg:SeVNAlg}
\begin{algorithm}
\label{alg:SeVNAlg}
\caption{survivable embedded virtual network request embedding algorithm}
\begin{algorithmic}[1]
\REQUIRE $G (V,E)$: virtual network's request; $G (S,L)$: substrate network .
\ENSURE Generate SeVN and embed SeVN augment resource into substrate network.
\STATE embed Virtual Network $G^V$ into Substrate Network $G^S$\cite{liu2011completing}.
\STATE extract embedded virtual network eVN $G\left( {\hat S,\hat L} \right)$ from SN corresponding to this VN embedding request
%\STATE AllMinimumCycle($G(V,E)$)
\FORALL{$v_i$ such that $v_i\in G\left( {\hat S,\hat L} \right)$}
\STATE Decompose eVN $G\left( {\hat S,\hat L} \right)$ into two star structure sets $VirtualStar(v_i)$ and $PhysicalStar(s_j)$ from graph $G\left( {\hat S,\hat L} \right)$.
\STATE Construct items based star structure set $VirtualStar(v_i)$.
\STATE Construct knapsacks based star structure set $PhysicalStar(s_j)$ from embedded virtual network eVN.
\STATE Construct edge cost matrix based in (\ref{eq:new edge weight}).
\STATE Solve multiple knapsack problem through dynamic programming in Sec.\ref{lab:DynamicProgrammingEquation}.
\STATE Add new nodes, connect new edges, re-allocate node computing and edge's bandwidth into $G\left( {\hat S,\hat L} \right)$ to construct new graph $G\left( {\hat S,\hat L} \right)$.
\ENDFOR
\STATE Embed augmented resource or startup new nodes from SeVN $G\left( {\hat S,\hat L} \right)$ into substrate network $G(S,L)$
\end{algorithmic}
\end{algorithm}

%1 node+2 node computaion+12 bandwidth
%0 node+1 node computaion+5 bandwidth
%1 node+5 node computaion+11 bandwidth
%1 node+6 node computaion+3 bandwidth
%=3 node+14 node computaion+31 bandwidth
%
%1 node+2 node computaion+12 bandwidth
%0 node+1 node computaion+5 bandwidth
%0 node+2 node computaion+3 bandwidth
%1 node+6 node computaion+3 bandwidth
%=2 node+11 node computaion+23 bandwidth
