\section{Related work}
\label{sec:RELATED_WORK}
As discussed in introduction, virtual network embedding (VNE) is the central resource allocation problem in network virtualization. It deals with the efficient mapping of virtual networks onto physical network resources. More specifically, for each virtual network creation request, VNE is responsible for mapping virtual nodes onto physical nodes and virtual edges onto one or more physical paths.

The VNE problem, with constraints on virtual nodes and virtual links, \note{We have constraints for physical resources. What do you mean constraints for virtual nodes and edges?} can be reduced to the NP hard multi-way separator problem \cite{yu2008rethinking}, even if the schedule of VN requests is known beforehand \cite{herrera2016resource}. In order to reduce the hardness of the VN embedding problem and enable efficient resource allocation, several heuristic virtual network embedding algorithm are proposed.  However, \rev{these algorithms \cite{yu2008rethinking,chowdhury2009virtual,houidi2008distributed,lischka2009virtual,cheng2011virtual,botero2012energy,butt2010topology,fajjari2011vne,chowdhury2010polyvine,guo2011shared,zhu2006algorithms} generally assume the physical network is operational all the time. The virtual network will stop functioning or suffer from poor performance upon the failures of physical nodes.}
%which is not realistic. The existing heuristics are not capable of handling physical node failures, which may lead to poor performance and increased frustration for the service provider.

\rev{Very limited \cite{yu2011cost,yeow2010designing,rahman2010survivable,koslovski2010reliability,yu2017survivable,patel2014survivable} are made to solve the survival virtual network embedding problem. One approach} is to first build the redundant virtual network, then map the virtual network to the physical network. \rev{As one extreme case \cite{patel2014survivable}, 1+1 protection is assumed in T-SDN, where dedicated backup resources are provisioned for} each virtual node and virtual link in a VN request.\rev{ Although simple to implement, it requires a} large amount of backup resource. To reduce the resource cost, \cite{yeow2010designing,yu2011cost} proposes two approaches to first enhance the \rev{$n$-node} virtual network to 1-redundant and $k$-redundant \rev{virtual network with $n+1$ and $n+k$ nodes in addition to} an appropriate number of redundant virtual links. Compared with 1+1 protection, $n+1$ and $n+k$  \rev{reduce the cost by sharing node resources among all virtual nodes.} \del{in the request,} \rev{However, this may not be feasible} in practical network virtulization scenarios in which not all virtual functions can be executed and supported by the same type of  physical nodes. As a result, using  1 virtual node  or $k$ virtual nodes without considering the virtual function type cannot guarantee the feasibility of the virtual network embedding. \note{But it seems that it is not difficult to extend this to the different types case.}

Different from \cite{patel2014survivable,yeow2010designing,yu2011cost}, \cite{guo2014survivable} proposes a survivable virtual network embedding approach \rev{without building} the redundant virtual network. However, similar to  \cite{rahman2013svne},  it can only work well in the network scenario without the function type constraint.

\rev{Besides the lack of} considering the function location constraint, \note{You didn't mention the location constraint so far.}  current works \cite{rahman2010survivable,koslovski2010reliability,yu2017survivable,patel2014survivable} often have a strong assumption that \rev{simultaneous failures from multiple physical nodes are mutually independent \cite{yeow2010designing}, and only considers a single node failure.}


To the best of our knowledge, we are the first to study the low cost survivable network embedding problem in a practical way by simultaneously considering the node capacity constraint, link bandwidth constraint, and network function type constraint. \rev{We will also show that our algorithm can} work well in network scenarios with both single and  multiple \rev{node failures.}
