\section{Related work}
Network virtualization is a promising technology to reduce the operating costs and management complexity of networks, and it is receiving an increasing amount of research interest\cite{chowdhury2009network}. Survivability is bound to become a more prominent issue as infrastructure providers move toward virtualizing their networks over cheaper commodity hardware\cite{bhatia2008trellis}. we are concerned with critical virtual nodes and embedding them as an entire infrastructure with Survivability guarantees. Fault tolerance is provided in data centers \cite{guo2009bcube} through excessive redundant nodes and links organized in a special way. These works provide Survivability but do not customize Survivability guarantees to embedded VInfs. However, such slice is not used as a back-up, but as a monitoring tool, and as a way to debug
the network in the case of failure.\cite{wang2008virtual} considered the use
of virtualized router as a management primitive that can be used to migrate routers for maximal reliability.

\subsection{Types and characteristics of failures} Survivable virtual
network embedding deals with failures in the substrate and
virtual network. The challenges to be considered are link and
node failures, which have to be backed up before the failure
or recovered after failure. Failures can occur at different layers
in the network. For example at the physical layer, a fiber cut
may cause a physical dis-connectivity. In\cite{markopoulou2004characterization}, it is shown
that 20 \% of all failures in an IP backbone are resulting from
maintenance activities. About 53 \% of the unplanned link
failures are due to router-related \cite{markopoulou2004characterization}. In a network, single
and also multiple failures can occur. The single failure case
happens more often than multiple simultaneous failures. The
study \cite{markopoulou2004characterization} states that about 70 \% of the unplanned link failures
are single link failures. A study \cite{gill2011understanding} about network-related
failures in data centers found out that link failures happen
about ten times more than node failures per day. Usually node
failures are due to maintenance \cite{gill2011understanding}.

\subsection{Survivable failure methods} There are two main survivability methods: protection and restoration \cite{ramamurthy2003survivable}. Failure protection is done in a proactive way to reserve the backup resources before any failure happens. Reactive mechanisms, which are called restoration mechanisms, react after the failure occurs and start the backup restoring mechanism. However, some data loss is possible in the reactive case. There exist two kinds of backups for the protection scheme: dedicated backup or shared backup. In shared backup, the resources for the backup may be shared with other backups. In the dedicated case the backup resources are not shared for other backups.

Meanwhile some works address node fault tolerance at the virtualization level. Bressoud \cite{bressoud1996hypervisor} was the first few to introduce fault tolerance at the hypervisor. Two virtual slices residing on the same physical node can be made to operate in sync through the hypervisor on the same physical node. Others \cite{wang2008virtual,cully2008remus} have made progress for the virtual slices to be duplicated and migrated over a network. Various duplication techniques and migration protocols were proposed for different types of applications (web servers, game servers, and benchmarking applications) \cite{wang2008virtual,cully2008remus} and Kemari \cite{tamura2008kemari} are two other systems that allows for state synchronization between two virtual nodes for full, dedicated
redundancy. However, these works focus on the practical issues, and do not address the resource allocation issue. VNsnap \cite{kangarlou2009vnsnap} is another method to take static snapshots of an entire virtual infrastructure to some reliable storage, in order to recover from failures. This can be stored as reliably
in a distributed manner as replicas \cite{chang2008bigtable}, or as parities \cite{dimakis2006decentralized,yeow2011highly}. VNsnap does not address synchronization, nor guarantee sufficient resources for recovery from snapshots. Fundamentally, there are methods to construct topologies for redundant nodes that address both nodes and links reliability \cite{ajtai1992fault,dutt1997node}. Based on some input graph, additional links are introduced such that the least number is needed. However, a node failure, in this case, may involve migrations among the remaining nodes to preserve the original topology. This may not be suitable in a scenario where migrations
may disrupt other running parts of the network.

%Our problem involves virtual network embedding [8, 18]
%with added node and link redundancy for reliability. Our
%model employs the use of path-splitting [26], which allows a
%link to be split over multiple routes such that the aggregate
%flow across those routes equal to the demand between the
%two nodes. This gives more resilience to link failures and
%allows for graceful degradation. A related work that does
%not use path-splitting for embedding reliability is [16].

%Zhang et al. [7,8] have considered substrate resource sharing among multiple virtual networks. However, their method only con- siders substrate resource sharing within the same priority class and does not consider sharing among different priority classes while at the same time satisfying the different latency require- ments. Thus, the method cannot be applied to the VNE problem where substrate resource sharing among multiple priority classes is required within each requested virtual network. This paper pro- poses a heuristic VNE method to minimize the required amount of substrate resources due to fair substrate resource sharing among multiple virtual networks, while considering the existence of mul- tiple priority classes that share the substrate resources with one another within each virtual network. Since substrate resource shar- ing among multiple virtual networks is expected to occur on the substrate link bandwidth more frequently, the proposed heuris- tic method prioritizes virtual link assignment rather than virtual node assignment, in contrast to most of the existing heuristic methods.

The organization of this paper is as follow. In the next section, we briefly describe the background, notations and define survivability in Sec.\ref{sec:ProblemFormulation}. Finally, we evaluate and validate the ideas through simulation in..
