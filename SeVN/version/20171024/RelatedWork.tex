

Network virtualization is a promising technology to re-
duce the operating costs and management complexity of net-
works, and it is receiving an increasing amount of research
interest [7]. Reliability is bound to become a more promi-
nent issue as infrastructure providers move toward virtual-
izing their networks over cheaper commodity hardware [3].
Analysis on the reliability of overlay networks in terms of
connectivity in the overlays has been developed [17]. Un-
fortunately, it is not applicable to our problem as we are
concerned with critical virtual nodes and embedding them
as an entire infrastructure with reliability guarantees.
Fault tolerance is provided in data centers [19, 13] through
excessive redundant nodes and links organized in a special
way. These works provide reliability but do not customize
reliability guarantees to embedded VInfs.
While we are not aware of works studying the allocation of
reliable virtual networks, [23] considered the use of “shadow
VNet”, namely a parallel virtualized slice, to study the re-
liability of a network. However, such slice is not used as a
back-up, but as a monitoring tool, and as a way to debug
the network in the case of failure. [22] considered the use
of virtualized router as a management primitive that can be
used to migrate routers for maximal reliability.
Meanwhile some works address node fault tolerance at the
virtualization level. Bressoud [4] was the first few to intro-
duce fault tolerance at the hypervisor. Two virtual slices
residing on the same physical node can be made to operate
in sync through the hypervisor on the same physical node.
Others [10, 9] have made progress for the virtual slices to be
duplicated and migrated over a network. Various duplica-
tion techniques and migration protocols were proposed for
different types of applications (web servers, game servers,
and benchmarking applications) [9]. Remus [10] and Ke-
mari [21] are two other systems that allows for state syn-
chronization between two virtual nodes for full, dedicated
redundancy. However, these works focus on the practical
issues, and do not address the resource allocation issue.
VNsnap [15] is another method to take static snapshots of
an entire virtual infrastructure to some reliable storage, in
order to recover from failures. This can be stored as reliably
in a distributed manner as replicas [6], or as parities [11,
25]. VNsnap does not address synchronization, nor guaran-
tee sufficient resources for recovery from snapshots.
Fundamentally, there are methods to construct topologies
for redundant nodes that address both nodes and links reli-
ability [1, 12]. Based on some input graph, additional links
are introduced such that the least number is needed. How-
ever, a node failure, in this case, may involve migrations
among the remaining nodes to preserve the original topol-
ogy. This may not be suitable in a scenario where migrations
may disrupt other running parts of the network.

Our problem involves virtual network embedding [8, 18]
with added node and link redundancy for reliability. Our
model employs the use of path-splitting [26], which allows a
link to be split over multiple routes such that the aggregate
flow across those routes equal to the demand between the
two nodes. This gives more resilience to link failures and
allows for graceful degradation. A related work that does
not use path-splitting for embedding reliability is [16].