
\section{Introduction}
why network virtualization is significant.

network virtualization has gained significant attention in recent years as an effectivemeans to share substrate network (SN) infrastructure among several virtual network (VN) service providers so as to improve the utilization of the substrate resources \cite{armbrust2009above,yu2008rethinking}. In the network virtualization environment, provisioning a VN is a promising way to support distributed applications and services which require the coordination of multiple geographically distributed facilities (e.g. storage arrays or computer clusters). With the maturity of the optical network technology and their high-bandwidth and low-latency characteristics, we will be able to deploy a federated computing and networking system (FCNS) \cite{zhang2013effective,papagianni2013optimal}, which interconnects a large number of facility nodes with optical network managed or controlled integratedly, to enable such large scale distributed applications.

what situation happen

In the multi-tenant network virtualization environment, one challenging problem raised is how to efficiently embed VN requests with various constraints. This is of utmost importance for increasing the utilization of SN resources and infrastructure providers’ revenue. To establish the VN, one needs to embed task nodes onto facility nodes, and virtual edges onto one or more substrate paths\cite{yu2008rethinking} , as well as allocating sufficient substrate resources which should not violate the computing/ bandwidth capacity limits of facility node/substrate link.

what is VN

In network virtualization, the primary entity is the Virtual Network (VN). A VN is a combination of active and passive network elements (network nodes and network links) on top of a Substrate Network (SN). Virtual nodes are interconnected through virtual links, forming a virtual topology. By virtualizing both node and link resources of a SN, multiple virtual network topologies with widely varying characteristics can be created and co-hosted on the same physical hardware.



why node failure happern
With infrastructure rapidly becoming virtualized, shared and dynamically changing, it is essential to have strong reliability support from the physical infrastructure, since a single physical server or link failure affects several shared virtualized entities. Providing reliability is often linked with.over-provisioning both computational and network capacities, and employing load balancing for additional robustness. Such highly reliable systems are good for applications where large discontinuity may be tolerable, e.g. restart of network flows while re-routing over link or node failures, or partial job restarts at node failures. A higher level of fault tolerance is required at applications where some failures have a substantial impact on the current state of the system. For instance, virtual networks with servers which perform admission control, scheduling, load balancing, bandwidth broking, AAA or other NOC operations that maintain snapshots of the network state, cannot tolerate total failures. In master slave/ worker architectures, e.g. MapReduce, failures at the master nodes waste resources at the slaves/workers.

With network infrastructure rapidly becoming  virtualized, shared and dynamically changing, it is essential to have strong reliability support from the physical infrastructure, since a single physical server or link failure affects several shared virtual entities. Providing reliability is often linked with over-provisioning both computational and network capacities, and employing load balancing for additional robustness. However, we do not talk about the capacity requirement and geographical location requirement of virtual node, and the bandwidth capacity requirement of virtual link in this paper.


why survivable VN is paramount

However, with network virtualization gaining momentum, the survivability challenges in VNE should also be well investigated. In a large networked computing system, hardware and software failures of facility nodes and communication resources (e.g., links and switching nodes) are norm instead of exception, such as power outages caused by virus attack, disk failures, misconfiguration or fiber cut \cite{xu2012survivable,rahman2010survivable,rahman2013svne,guo2011shared,chen2010resilient}. Such failures will force the virtual node (links) assigned to it to be migrated/re-embeded to another facility node at a geographically different location (link disjoint substrate path). This means that, in order to survive from the disruptions due to such failures, one must reserve redundant facility nodes and bandwidth on fiber links such that after any failure, there are adequate remaining computing and networking resources to migrate/remap the VN request. Accordingly, the problem of minimizing the resources, including computing (CP.) and communication (CM.) resources, reserved for VN request to tolerate substrate failures, (hereafter called Survivable
VNE problem, SVNE) is both critical and challenging. Actually, SVNE problem is quite different with the protection approach
in IP over WDM network by investigating facility node failure caused virtual node failure problem and employing node migration induced VN remapping as an unique recovery strategy.

shared resource

To survive a substrate link failure, pre-computed alternative paths inVN are used in general and the bandwidths are allocated before or after a failure. For instance, the reactive detour solution is employed after a substrate link failure in \cite{rahman2010survivable}, while authors in \cite{rahman2013svne,guo2011shared} propose a proactive backup approach (considering the backup resources sharing) to avoid service disruption in reactive restoration approach and improve the substrate resource utilization.


FD  FI

In terms of SVNE capable of recovering/re-embedding the task nodes after a facility node failure, there are two basic approaches: Failure Dependent Protection (FDP \cite{yu2010survivable} and Failure Independent Protection (FIP) \cite{yeow2011designing}, and the differences between them are as follows. In FIP, a host (facility) node is assigned and dedicated to backup all working host (facility) nodes. That is, no matter which working host node fails, the affected task node will be migrated to the only one backup host node. On the other hand, with FDP, each working host node can have a different backup host node under different failure scenarios. In fact, after a failure, even an unaffected task node may be migrated from a working host node to its corresponding backup host node, as a result of re-embedding the entire task graph. In other words, FDP could provide more flexibility in survivable VN designing by allowing task nodes migrating freely after failure, so FIP could be considered as a special case of FDP and FDP is expected to use fewer resources at the cost of more task nodes migrations after a failure.


serive type

Through synchronization\cite{bressoud1996hypervisor,cully2008remus} and migration techniques\cite{clark2005live,wang2008virtual} on virtual machines and routers, we suppose that every virtual network have specific service functions, in addition that fault tolerant can be introduced at the virtualization layer\cite{yeow2011designing}.

node cost larger than edge cost

As nodes often represent expensive components (servers) and edges represent less expensive interconnections (links), most attention has been devoted to node faults, i.e., the removal of nodes (and their incident edges), rather than edge faults where only edges are removed.

related work
Through synchronization [4, 10] and migration techniques [9, 22] on virtual machines and routers, we postulate that
fault tolerance can be introduced at the virtualization layer. This has several benefits. Different levels of reliability can
be customized and provisioned over the same physical infrastructure. There is no need for specialized, fault tolerant
servers. Instead, redundant (backup) virtual servers can be created dynamically, and resources are pooled together, in-
creasing the primary capacity. Both will lead to a better overall utilization of the physical infrastructure.


specify time, we can protect one embedded virtual network and which nodes could be protected.

Our paper focus on the problem of resource deployment for virtual network infrastructure with reliability guarantee.

we focus in this paper on independent node failures

The organization os this paper is as follow. In the next section, we briefly describe the background, notations and define reliability in Sec.\ref{sec:ProblemFormulation}. Finally, we evaluate and validate the ideas through simulation in
%
%\begin{table}
%\centering
%\begin{tabular}{cc}
%\hline
%Term & Description \\
%$G(V,E,S)$ & $G(V,E,S)$ denotes the Virtual Network, consisting of nodes V, links E and service functions\\
%$B(V,S)$ & $B(V,S)$ denotes the Backup nodes set, consisting of nodes V and service functions S which is corresponding to every nodes \\
%\hline
%\end{tabular}
%\caption{terminology used throughout this paper}\label{tab:term}
%\end{table}

