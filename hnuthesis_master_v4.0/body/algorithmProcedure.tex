
\subsection{算法步骤}
在本节中,我们首先介绍我们的完整解决方案,然后分析其复杂性。

算法1显示完全min-min srg不相交。路由算法算法中的输入参数包括网络图(G)、源(S)、目标(D)应将包含链接集包括在AP(I)中,以及排除链路集不应包含在AP(O)中。算法的输出是SRLG不相交路径对。(美联社,BP)。

Algorithm \ref{alg:min-min} shows the complete min-min SRLG disjoint routing Algorithm. The input parameter in the algorithm includes  the network graph ($G$), the source ($s$), the destination  ($d$), the inclusion link set should be included in AP ($\mathbb{I}$), and the exclusion link set should not be included in AP ($\mathbb{O}$). The output of the algorithm is the SRLG disjoint path pair $(AP,BP)$.


寻找SRLG不相交的路径,最小的首先搜索网络中的权重ap在步骤2中找到AP(G,s,d,i,O),其中i=φ,O=φ,然后通过查找srlg不相交的bp对bp进行搜索。(g,s,d,AP)第4 步。特别是,AP路径可以是通过Dijkstra算法发现的。为了计算BP,发现SRLG不相交BP(G,s,d,AP)包括两个步骤。首先,对于AP上的所有链接,删除共享这些链接的共同风险。第二,Dijkstra算法再次运行网络的其余链接来计算。从s到d的第二条最短路径BP。


To look for the SRLG disjoint paths, the smallest weight AP in the network is searched  first through FIND\_AP$(G,s,d, \mathbb{I},\mathbb{O})$ in step \ref{alg:findap} with $\mathbb{I}=\phi,\mathbb{O}=\phi$, and then BP is searched through  FIND\_SRLG\_Disjoint\_BP
$(G,s,d,AP)$ in step \ref{alg:findsrlgdisjointbp}. Specially, the AP path can be found through the Dijkstra algorithm. To calculate BP, FIND\_SRLG\_Disjoint\_BP
$(G,s,d,AP)$ includes two steps. First, for all the links on the AP, remove the links that share a common risk with these links. Second, the Dijkstra's algorithm runs over the remaining links of the network again to compute the second shortest path BP from $s$ to $d$.


如果我们能找到一个srg不相交的bp,那么Min-Min srlg解决了不相交的路由问题,找到了路径对。返回,如步骤6所示。否则,陷阱问题发生了。要处理陷阱问题,步骤8首先找到SRLG冲突链接集T,步骤10进一步划分原Min-Min SRLG不相交路由问题到T子-基于冲突集的问题T.所有的子问题都可以并行执行。步骤11利用集合F存储满足APi 6=φ和BPI 6=φ的可行解。在F中的所有可行解中,具有选择最低的AP路径权重作为最优解。对于原Min-Min SRLG不相交路由问题


If we can find a SRLG-Disjoint BP, the Min-Min SRLG disjoint routing problem is solved and  the path pair found is returned as shown in step \ref{alg:returnpathpair}. Otherwise, a trap problem happens. To handle the trap problem, step  \ref{alg:findsrlgconflictinglinkset} first finds the SRLG Conflicting Link Set $\mathbb{T}$, step \ref{alg:dividedandconquer} further divides the original  Min-Min SRLG disjoint routing problem into $\left| \mathbb{T} \right|$ sub-problems based on the conflict set $\mathbb{T}$. All the sub-problems can be executed in parallel. Step \ref{alg:findfeasible} utilizes the set $\mathbb{F}$ to store the feasible solutions  satisfying that  both $A{P_i} \ne \phi$ and $B{P_i} \ne \phi$. Among all the feasible solutions in the $\mathbb{F}$, the path pair with the lowest AP path weight will be selected as the optimal solution for the original Min-Min SRLG disjoint routing problem.


我们以图3中的图表为例来说明算法1.根据步骤2,我们的算法首先搜索对于路径AP={e1,e2,e3,e4,e5,e6,e7,e8}具有最小的通过Dijkstra算法加权,路径显示为图3(C) 中的虚线。删除AP 上的链接后此外,与AP有共同风险的链接,我们有图3(D)中的图,这是一个不连通图,找不到BP。然而,如图3(B)所示,拓扑中存在SRLG-不相交路径对.因此,陷阱问题就会发生。在找到SRLG 冲突链接之后设置{e2,e5,e6}在步骤8中,我们对将原问题P(∅,∅)分为三个子问题,P(∅,{e2}),P({e2},{e5})和P({e2,e5},{e6})第10步。并行执行这些子问题之后,返回具有最小AP权重的路径对。


We take the graph in Fig.\ref{fig:CompositeGraph} as an example to illustrate  our Algorithm \ref{alg:min-min}.
According to step $\ref{alg:findap}$, our  algorithm first searches for a path  $\mathbb{AP}=\{e_1,e_2,e_3,e_4,e_5,e_6,e_7,e_8\}$ with the smallest weight through the dijkstra algorithm, with the path shown as the dotted line in Fig.\ref{fig:CompositeGraph}(c). After removing the links on AP and also the links that share the common risks with AP, we have the graph in Fig.\ref{fig:CompositeGraph}(d), which is a disconnected graph and no BP can be found. %\note{You gave the exact example earlier. Dear sister, from the previous comments, we find we need a example to illustrate the how our algorithm runs}
However, as shown in Fig.\ref{fig:CompositeGraph}(b), there exists a SRLG-disjoint path pair in the topology. Therefore, a trap problem happens. After finding the SRLG conflict link set $\{e_2,e_5,e_6\}$ in step $\ref{alg:findsrlgconflictinglinkset}$, we apply divide and conquer to divide the original problem ${\mathcal P}(\emptyset ,\emptyset )$  into three sub-problems, ${\mathcal P}(\emptyset ,\{e_2\} )$, ${\mathcal P}(\{e_2\} ,\{e_5\} )$ and ${\mathcal P}(\{e_2,e_5\} ,\{e_6\} )$ according to step $\ref{alg:dividedandconquer}$. After executing these sub-problems in parallel, the path pair with the minimum AP weight is returned.

 %For example, subproblem ${\mathcal P}(\{e_2\} ,\{e_5\} )$ obtain a AP ($e_1,e_2,e_3,e_{11},e_{12},e_8$) whose corresponding BP ($e_15,e_18,e_3,e_6,e_7,e_{14}$). compare the AP's weight of ${\mathcal P}(\{e_2\} ,\{e_5\} )$  with other parellel subproblems in order to obtain optimal solution. subproblem ${\mathcal P}(\emptyset ,\{e_2\} )$ obtain a AP ($e_9,e_3,e_4,e_5,e_6,e_7,e_8$) which result finding no BP, so continue divide and conquer subproblem ${\mathcal P}(\emptyset ,\{e_2\} )$ as like Alg.\ref{alg:min-min} step 8.}

\begin{algorithm}
\small{
\caption{Min-Min}
\begin{algorithmic}[1]
\label{alg:min-min}
%\caption{Main process of Algorithm}
\REQUIRE
$G$: the network graph\\
$s$: the source node\\
$d$: the destination node \\
$\mathbb{I}$:   the inclusion link set should be included in AP\\
$\mathbb{O}$: the exclusion link set should not be included in AP\\
\ENSURE
AP: the active path\\
BP: the backup path
\STATE $AP=\emptyset$, $BP=\emptyset, \mathbb{I}=\emptyset, \mathbb{O}=\emptyset$
\STATE $AP\leftarrow$ FIND$\_$AP$(G,s,d,\mathbb{I},\mathbb{O})$\label{alg:findap}
\IF{$AP\neq\emptyset$}
    \RETURN $BP\leftarrow$ FIND\_SRLG\_Disjoint\_BP$(G,s,d,AP)$\label{alg:findsrlgdisjointbp}
    \IF{$BP\neq\emptyset$}
        \RETURN {path pair $(AP,BP)$}\label{alg:returnpathpair}
    \ELSE
        \STATE find SRLG Conflicting Link Set $\mathbb{T}$\label{alg:findsrlgconflictinglinkset}
        \STATE $\mathbb{T}\leftarrow \mathbb{T}-(\mathbb{I}\cup\mathbb{O})$
        %\IF{$\mathbb{T}\neq \emptyset$}
        \STATE {divide and conquer for execution in parallel\\
        \tiny{
        $\!\!\!\!\!\!\!\!\!\!\!\!\!\!\!\!\!\!\!\left\{ \begin{array}{l}
 \left( {A{P_1},B{P_1}} \right)={{Min-Min}}\left( {G,s,d,\mathbb{I} ,\mathbb{O}\cup\{ {t_1}\} } \right), \\
 \left( {A{P_2},B{P_2}} \right)={{Min-Min}}\left( {G,s,d,\mathbb{I}\cup\{ {t_1}\} ,\mathbb{O}\cup\{ {t_2}\} } \right), \\
 \left( {A{P_3},B{P_3}} \right)={{Min-Min}}\left( {G,s,d,\mathbb{I}\cup\{ {t_1},{t_2}\} ,\mathbb{O}\cup\{ {t_3}\} } \right), \\
  \cdots  \\
 \left( {A{P_{\left| \mathbb{T} \right|}},B{P_{\left| \mathbb{T} \right|}}} \right) = {{Min-Min}}\left( {G,s,d,\mathbb{I}\cup \{ {t_1},{t_2}, \cdots ,{t_{\left| \mathbb{T} \right| - 1}}\} ,\mathbb{O}\cup\{ {t_{\left| \mathbb{T} \right|}}\} } \right) \\
 \end{array} \right.$
 }
        }\label{alg:dividedandconquer}
        \STATE{  {$\!\!\!\!\!\!\!\!\!\!\!F\leftarrow$ FIND\_FEASIBLE$(( {A{P_1},B{P_1}} )),\cdots,( {A{P_{|\mathbb{T} |}},B{P_{| \mathbb{T} |}}} )$}}\label{alg:findfeasible}
        %\ENDIF
         \IF{$F\neq{\emptyset,\emptyset}$}
         \RETURN{path pair $(AP,BP)$ satisfying that $AP = \mathop {\arg \min }\limits_{AP} \left\{ F \right\}$}
        \ENDIF

    \ENDIF
\ENDIF
\end{algorithmic}
}
\end{algorithm}