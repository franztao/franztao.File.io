\chapter{虚拟网络映射问题}
\section{SDN网络虚拟化概述}
未来互联网架构提案可以分为两种主要方法:一元论和多元论。在一元论模型 中,网络有一个整体架构,它必须足够灵活,以便支持所有将会出现的应用。在 一元论这种网络架构方法中,每次仅有一个协议栈运行在物理基层之上。而多元 论是基于互联网必须同时支持多个网络的思想,每个网络都会运行协议栈,该I办 议栈是适合一项给定应用需求的。这种针对特定服务而创建特定网络的做法,简 化了诸如安全、移动性或服务质量等特性的新应用的部署,为提供不同服务而设 计多个网络的做法比为同时处理不同服务而设计一个独特网络的做法要更容易。 因此,多元论方法可以被理解为是基于“分而治之”方法的。由于寻找涵盖所有可能 需求的单一解决方案是非常困难的,构造带有支持这些需求的网络就成为了一种 可行的替代方法,这是多元论架构的主要特点,多元论更被接受的原冈是一元论 方法不仅必须解决所有己知的互联网问题,而且也要求在未来网络的演进过程中 依然能够运行。此外,多元论的另一个关键优势就是其固有的后向兼容性,因为当前的互联网是可以并行运行的。多元论提案是基于多个网络运行在同一基础设 施之上的同一思想,即使它们在报文格式、寻址方案和协议方面存在差异,也是 如此,因为这些网络丼享一个物理层,例如路由器和链路等,所以对基础介质的 多种访问方式必须由同一个检测器进行编排。这个检测器就是一个特殊用途的软 件,它将共享介质虚拟化后提供给运行在其上的多个网络因此,每个网络的运 行,就像它正在使用给定的物理资源一样,这些网络就是虚拟网络,但是为了能 够共享一个物理底层,它们不得不面临并且要处理性能的挑战。

虚拟化是以软件模拟硬件平台如服务器、存储设备或网络资源的能力,所有 功能都与硬件分离,并被模拟为“虚拟实例”,具有像传统的硬件解决方案一样的操 作能力。此外可以使用单个硬件平台来支持多个虚拟设备或机器,这些虚拟设备 或机器根据需要容易变动,因此,与传统的基于硬件的解决方案相比,虚拟化解 决方案的可移植性,可扩展性和成本效益会更高。

计算机虚拟化主要是用在数据中心,可以在一台物理机上同时运行几台服务 器。网络的虚拟化与计算机的虚拟化比较类似,支持网络虚拟化共享的资源是网 络。将虚拟化技术应用到网络的最主要动机是尽量在每个虚拟网络也即虚拟分片 内运行各自定制的协议栈m。在计算机中,网络虚拟化是将硬件和软件的网络资源 和网络功能组合成单个基于软件的管理实体。网络虚拟化包括平台虚拟化,通常与资源虚拟化相结合。另外,网络虚拟化通常分为外部虚拟化和内部虚拟化。外 部虚拟化将网络或网络的部分组合成虚拟单元,将一个或多个局域网(LAN)组合为 虚拟N络,可以提高大型网络或数据中心的效率,虚拟局域网(VLAN)和网络交换 机包括关键组件。使用此技术,系统管理员可以将在物理上连接到同一本地网络 的系统配置为单独的虚拟网络。相反,管理员可以将单独的局域网(LAN)上的系统 组合成跨越大网络段的单个VLAN。而内部虚拟化是为单个网络服务器上的软件容 器提供类似网络的功能,使用软件容器或伪接口配置单个系统,以使用软件仿真 物理网络,这可以通过将应用程序隔离到单独的容器或伪接口来提高单个系统的 效卓。

在软件测试中,软件开发人员可以使用网络虚拟化来测试在模拟网络开发环 境下的软件。作为应用程序性能工程的一个组件,网络虚拟化使开发人员能够在测试环境中模拟应用程序、服务、从属性和最终用户之间的连接,而无需在所有 可能的硬件或软件系统上测试软件。当然,测试的有效性取决于在仿真真实硬件 和操作系统的网络虚拟化的准确性。
软件定义网络现在也有多种,在软件定义光传送网络(Software Defined Optical Network,SDON)架构中,SDN控制器可以用于网络虚拟化和自动控制,在SDN控 制器上的通用多协议标签交换协议(Generalized Multiprotocol Label Switching, GMPLS)已经进行了虚拟化,可以运行控制平面不同的网络切片,链接层发现协议 (Link Layer Discovery Protocol, LLDP)会及时捕获拓扑变化,这种架构可以在全局 网络视图内优化算法[1'虽然SDN是重建互联网架构的有效方法,但是在SDN控 制器中网络协议程序是混杂在一起的,在相同SDN控制器上运行的每个网络协议, 都需要有独立的网络环境和各自私有的资源,所以可以在SDN网络操作系统上设 计一个虚拟化云平台[11]。

数据中心处的网络需要虚拟化,就像计算和存储的功能一样,这样可以很方 便地隔离共存的多个租户。现有的数据中心处的网络虚拟化经历了两个阶段:第 一阶段是使用简单的VLAN和MAC地址来实现隔离;第二阶段的是使用IP覆盖网 络[1'这两个阶段的方法都有各自的缺陷,首先,基于VLAN和MAC地址的方法 难以管理,并且将虚拟机网络与物理基础设施直接绑定会使得虚拟机的放置和移 动不灵活。其次,IP覆盖网络对参与虚拟机网络的虚拟机数量有限制并且出现问题 后难以调试。现在提出了一种新的方法,利用多层标签来实现隔离并且可以指定 通过数据屮心网络的路由,分组标签可以使用SDN控制平面,控制平面实现了数 据中心交换机的OpenFlow控制。


当虚拟化应用于网络时,虚拟化可以创建硬件和软件网络资源(交换机、路由器等)的基于软件的逻辑视图。物理网络设备简单地负责分组的转发,而虚拟网络 提供智能抽象,这样有利于部署以及管理网络服务和底层网络资源。因此,网络 虚拟化可以调整网络以更好地支持虚拟化环境。


\section{网络虚拟化环境}
虚拟网络的概念并不新颖,我们的生活中早就出现丫虚拟专用网VPN和虚拟 局域网VLAN,并且可编程虚拟化路器也是当前和未来构建互联网实验床的核心设 备。
网络虚拟化抽象了传统上通过硬件传递到逻辑虚拟网络中的网络连接和服务, 该逻辑虚拟网络与管理程序中的物理网络分离并独立地运行。除了第二层和第三 层服务,如交换和路由,网络虚拟化通常虚拟化第四层到第七层服务,包括防火 墙和服务器负载均衡|131。网络虚拟化可以解决当今数据中心的许多网络挑战,帮 助组织按需集中编程和配置网络,而无需直接接触底层基础设施。通过网络虚拟 化,组织可以扩展和调整工作负载和资源以满足不断发展的计算需求。另外,网 络虚拟化吋以克服内核化,此技术允许多个虚拟网络在共享物理基础架构上运行, 其中的关键步骤在于将虚拟网络请求映射到底层物理网进行资源的有效分配。

在常规的网络架构中,有相互独立的控制平面和数据平面,因此有两种基本方法可以对一个网元来实施虚拟化:(1)仅对控制平面进行虚拟化,也就是虚拟 层仅存在于数据平面和控制平面之间,在这种情形下,任意一个虚拟网络单独运 行自己的控制软件,但是是共享数据平面的,相比于常规的网络架构,这种虚拟 化方法可以在很大程度上改善网络的编程性能,原因在于它可以支持多个特定的 协议栈而非系统默认单一的协议;(2)同时对数据平面和控制平面进行虚拟化, 在这种情形下,除控制平面外,每个虚拟网络都可以实现自己的数据平面,这种 虚拟化方法更加改善了网络的编程性能,但是数据平面的虚拟化牺牲了报文转发 性能,因为它不再专用于一项共同的服务[141。在第一种虚拟化方法中,仅仅通过 对控制平面虚拟化就可以将虚拟化方法分成多种,这主要取决于各虚拟网络间数 据平面的隔离等级,如果是强隔离,那么每个控制平面仅仅可以访问自己的数据 平面分片,而不能去干扰其他的分片,如果数据平面在虚拟控制平面之间是共享 的,那么虚拟控制平面之间会出现互相干扰的情况。

在N络虚拟化环境中,多个异构的虚拟网络架构共享底层物理网设施,这种 共享是对底层物理网络设施进行抽象、隔离而形成的。不同的虚拟网络在很多方 面都是不同的:网络拓扑、服务的提供、技术的使用等。网络虚拟化主要是想要 极大地发挥资源共享的优势,进行灵活的配置与管理,支持未来网络发展的渐进部署。相比于现有的互联网,这种体系架构优劣分明:它提供了更灵活的资源供 使用,确保了服务质量和安全性;但与此同时,我们也需要设计一套更加复杂和 完善的虚拟网络管理机制来获得这种性能改善。
在虚拟网络中有三个角色:基础设施提供商、服务提供商和客户。每个角色 都有不同的目标:基础设施提供商通常专注于平衡负载,最大化收入和最小化成 本;服务提供商通常关注最大化收入,例如,尽可能多地支持客户或请求,他们 是客户和基础设施提供商之间的中介;客户通常专注于服务质量,例如下载时间 或比特率。也就是说,网络虚拟化的一个优点是虚拟网络可具有异构资源以满足 各种定制需求,虚拟化也可以提供实验环境为研宄人员评估新的协议。当服务提 供商生成虚拟网络请求时,它可以为每个虚拟节点和链路指定资源。因此,当将 虚拟网络请求映射到底层网络中特定的物理节点和链路上时,其必须满足对虚拟 节点和虚拟链路的约束。

\section{网络虚拟化的技术特征}
网络虚拟化也是现在网络领域的一个研宄方向,总的来说,它具有如下几个 特征:
(1)	隔离性,底层物理网络、虚拟网络和控制平面之间要进行安全隔离,通 过共享底层物理资源,能够大大地提高计算能力,并且可以优化存储资源和网络 资源的使用率。网络虚拟化平台在进行资源整合的同时还必须提供隔离,同时也 要隔离底层物理网络和虚拟网络的地址。
(2)	独立性,网络虚拟化平台的运行不受网络硬件的限制,而且未来出现的 网络架构会更好地支持商品化和虚拟化,从而能够进一步改善云的成本代价。
(3)	可以很好地复制物理网络的服务模式,对于运行在任一物理环境下的任 一工作负载,网络虚拟化平台必须能够提供很好的支持,此外,对于现有的网络 服务,比如负载均衡,WAN优化以及ACL等,也必须要能够提供。
(4)	网络虚拟化要遒循计算领域虚拟化的模式,在计算领域,虚拟机能够被 作为一个软状态来进行处理,这是该领域虚拟化的一个主要特征,网络虚拟化也 必须要具有同样的灵活性。
(5)	兼容性,网络虚拟化平台要能够与任何的hypervisor平台兼容。
(6)	网络虚拟化要具有云的规模和性能,网络虚拟化要有能力支持大规模的 网络部署环境,这样不仅可以允许存在大量的租户,而且还可以支持很多服务, 如数据中心的租用等。网络虚拟化还不应该允许网络中存在故障点或拥寒点,并 且可以支持故障切换功能和多路由功能。

\section{虚拟网络映射算法}
在网络虚拟化的研宄过程中,虚拟网络映射问题是网络虚拟化研究过程中的 主要挑战之一。虚拟网络映射是在底层物理网络上寻找满足虚拟网络请求的基础 物理资源的过程115]。该映射主要是虚拟网络链路和节点的映射。虚拟网络的一个 节点能够被映射到物理网络拓扑中的任意一个物理节点,并且一条虚拟链路也可 能会被映射到底层物理网的多条物理路径,因此,任意一个虚拟网络被映射到底 层物理网络时,都会有多种映射方案。所以怎么把业务供应商的虚拟网络请求与 物理网络进行映射就显得尤为重要。当虚拟网络的链路和节点有约束条件时,此 时的映射算法属于NP难问题,即使是考虑最简单的没有约束条件的情况,在虚拟 网络拓扑给定的情况下,也仍然是NP难问题。所以现在在大多的研宂中,基本是 利用启发式算法来求解虚拟网络的映射问题,尽量优化解决方案116]。
简单算法映射是虚拟网络映射相关问题的基础,没有考虑底层设备的故障和 复杂的基础设施构成,也没有考虑虚拟网络请求的在线达到。在将虚拟链路映射到底层物理路径上时,如果允许路径分割,那么一条虚拟链路将会映射到底层物 理网络的多条路径上,这些路径的源目的节点相同,并且在链路映射时,物理链 路需要有足够的带宽117]。在一次虚拟网络请求映射在物理网之后,物理网的剩余 节点的计算资源和链路的带宽容量将会相应地减少。
在虚拟网络的请求是已知的情况下,此时的虚拟网络映射问题可以转化为多路分离器(multi-way separator)问题求解,是一个NP难问题。静态虚拟网络映射算 法有两阶段映射算法和单阶段映射算法两种。两阶段虚拟网络映射算法是先进 行节点映射,在节点映射己经完成的情况下进行的链路映射研宄。其中链路映射 可以采取多种方案:最短路径、K最短路径、基于多商品流的映射等。单阶段虚拟 网络映射将节点映射和链路映射作为整体考虑,将两者同时进行来获得理想的映 射效果。典型的虚拟网络映射算法有:分布式协作分割算法、基于业务约束的映 射算法、嵌入虚链路的算法等。
以下图为例来说明简单的映射算法,将图2-l(a)所示虚拟网络请求的节点作为 扩展顶点加入到图2-l(b)所示的底层物理设施中,并且将物理节点和扩展顶点连接 形成扩展边,来构成如图2-2所示的映射扩展图。径中经过的扩展边就决定了相应虚拟节点的映射,因此虚拟网络的映射问题可以 转化为MIL(Mixed-integer 丨inear programming)问题。

在对虚拟网络请求静态映射时,可以根据全局拓扑和虚拟网络请求设计各种 不同的算法来达到最优性能,但是在现在的云计算还有数据中心的一些实际应用 中,用户需求一般是动态变化的。如分布式计算中,用户开始计算任务或者结束 计算任务,运行中的任务从计算密集阶段到通信密集阶段的过渡中,在游戏中加 入玩家或者减少玩家等,所有这些变化都会导致虚拟网络请求发生变化。与静态 映射算法中虚拟网络请求不随时间变化不同,动态映射算法根据实时达到的虚拟 网络映射请求对己映射的资源进行重映射,使得底层物理网络资源的使用率达到 最大化。由于动态映射在线实时的特性,它可将零碎的底层资源整合为易于管理 的资源,并且可以缓解网络中的节点和链路资源瓶颈,来确保网络的连通性[19]。
在实际的测量研宄中也发现,在网络中大量的流量呈现出动态变化的特性, 例如数据中心的流量。这类特性将会导致虚拟网络请求的计算容量和带宽资源也会动态变化。面对虚拟网络资源请求动态变化这一问题,现在还没有比较完美的 解决方法。也有算法提出用保守的资源映射策略来解决,也就是使得提供的资源 远远大于用户需求的资源来满足动态变化的特性,虽然用户需求满足了,但是这 种资源过配置(over-pro v isi oning)的策略,会浪费网络资源。
在动态映射过程中,若不要求对资源进行重配置,可以利用启发式算法和自 适应优化策略来提高性能,若要求在虚拟网络请求到达时对物理资源进行重新分 配,则可以设计一种选择性的虚拟网络重配罝方案,对虚拟网络请求进行优先级 排序,首先对优先极高的虚拟网络进行重映射[2G]。有许多启发式算法和贪婪算法 的目的是将虚拟节点合理地映射到底层物理节点以便最大化地利用资源,这种算 法一般是集中式地进行,需要有集中管理器来管理全局信息,利用全局信息,集中式映射算法可以避免资源分配时出现的冲突,并且可以很容易获得全局最优解。 但是,这种集中管理的机制会增大网络的开销,对于规模比较大拓扑动态变化的 网络,会使得算法的复杂度大大增加,并且当中央管理器出现故障时,会影响全 局网络。基于此,可以提出一种虚拟网络映射协议,设计一种分布式网络映射算 法,用于不同物理节点之间的通信和消息交换,来实现映射。
基于子图同态检测的虚拟网络映射算法是一种动态映射算法,该算法在同一 阶段同时对节点和链路进行映射。对于有容量约束以及在线到达等多种请求,该 算法都能解决,它比两阶段的方法更快,特别是在难以映射并且具有高资源消耗 的大型虚拟网络拓扑中,优势较为明显。
由于动态虚拟网络映射算法是•个更为复杂的NP难问题,一般考虑静态虚拟 网络映射算法。



