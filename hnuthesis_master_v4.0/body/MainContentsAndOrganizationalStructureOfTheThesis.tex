% !Mode:: "TeX:UTF-8"
\section{论文的主要研究内容和组织结构}
\subsection{论文的主要研究内容}
SDN作为新型网络架构,NFV作为未来网络研究的重要领域,两者的结合具有新的研究意义。在网络可生存性领域中,当网络故障发生时对业务主路径提供备份路径是一个重要基础研究领域,在网络虚拟化的过程中,不可避开的一个问题就是虚拟网络嵌入算法的研究。
\subsubsection{Min-Min SRLG 完全不相交路径对问题}
基于图论寻找SRLG完全不相交路径时,对陷阱问题提供了新的见解。具体来说,对于带有陷阱问题的AP,我们观察到AP路径上存在一组链路集,任何通过所有这些“问题”链路AP都不能找到一个SRLG完全不相交的BP。我们称之为SRLG冲突链路集。一旦遇到陷阱问题,建议寻找SRLG冲突链路集,而不是搜索所有可能的替代路径,基于最大流最小割定理\cite{ford2015flows}。 进一步提出了一种分而治之的min-min SRLG完全不相交路由算法,将原路由问题划分为多个子问题并行执行,找到可行的AP和BP对。研究的主要内容如下:
\begin{itemize}
  \item 通过巧妙地设置链路容量来构造一个新的流图,以便于发现SRLG 冲突链路集。
  \item 提出了一种寻找最小SRLG冲突链路集的算法,这有助于减少搜索替代SRLG完全不相交路径对的复杂性。
  \item 根据SRLG的风险共享的特性,将SRLG冲突链路集最小查找问题转化为子集覆盖问题,使我们能够应用通用算法在不同的复杂SRLG场景(包括一个或多个SRLG模式的链路)中寻找最小SRLG冲突链路集。
  \item 提出了一种新的分而治之算法,该算法能在遇到陷阱问题时将原有的min-min SRLG完全不相交路由问题划分为多个并行执行子问题。与现有技术相比,这样的解决方案搜索过程可以利用现有的AP搜索结果和并行执行来加快的路径查找。
  \item 在一个多核CPU平台上进行了广泛的仿真,以评估所提出的算法。仿真结果表明,该算法能够以更快的速度在不同的网络场景中找到最佳解。
\end{itemize}

\subsubsection{单物理节点故障可生存性虚拟网络嵌入问题}
嵌入算法主要考虑的是虚拟网络节点和链路向底层物理网的节点和链路的映射,考虑如何为虚拟网络在底层物理网中找到满足条件的最优嵌入,并且尽可能地提髙网络资源的利用率。%鉴于此,本文做了如下研究:
%\begin{enumerate}
%  \item 首先,研究SDN虚拟网络技术,对SDN网络虚拟化的环境进行了研究,并且探讨和总结了虚拟网络向底层物理进行嵌入时的不同嵌入算法。
%  \item 在虚拟网络向底层物理网络映射时,考虑节点和链路的映射以及路由算法的设计,在嵌入算法的设计中,假设节点己经映射并且节点具有功能类型的约束。分别设计算法实现此时提供可生存性的虚拟网络嵌入算法。
%\end{enumerate}

本课题将结合位置约束的概念,以虚拟网络请求的接受率、物理网络资源利用率和故障恢复率等指标作为算法性能评价指标,提出一个星型分解动态规划节点映射的可生存性虚拟网络嵌入算法。为了实现课题目标,已进行以下研究工作:
\begin{enumerate}
  \item 调研现有的可生存性虚拟网络映射算法,以及基于SDN的虚拟网络映射算法和可生存性虚拟网络映射算法,分析不同算法的优缺点,以及存在的共性问题。
  \item 提出一种星型分解动态规划节点映射的可生存性虚拟网络嵌入算法,提出虚拟星型图和物理星型图的概念,给出虚拟星型图与物理星型图之间节点映射的权值设定,转换节点映射的问题为多背包问题,通过动态规划方法解决这个多背包问题,理论分析了动态规划方法时空间复杂度。
  \item 设计和搭建仿真平台,对提出的嵌入算法进行仿真,并实现相关的对比方案。对实验结果进行分析,评估本文方案和对比方案的性能优劣得出,提出的算法比其他算法在接受率和资源利用率等指标上结果更优。
\end{enumerate}

\subsection{论文的组织结构}
本文依据网络可生存性、不相交路径以及虚拟网络嵌入等方面的研究现状与发展趋势,并结合自身对以上的探索和研究,将本文分为五章,各章主要内容如下,其由于硕士论文篇幅限制,本文有关不相交路径算法、虚拟网路嵌入算法和可生存性虚拟网络嵌入算法的总结将不在本文中,可在github网址\cite{Thesis}得到论文更详细版本:
\begin{itemize}
  \item 第一章为绪论部分,主要介绍了本课题的研究背景及其意义并且简要地分析了研究现状和发展趋势。
  \item 第二章介绍了网络可生存性相关原理,概述了网络的常见故障和网络故障恢复的两大机制,接着详细介绍了网络故障保护策略。
%  \item 第三章介绍了不相交路径的相关技术,概述了不同类型的不相交路径的问题,然后给出不同类型的不相交路径的约束条件,实际应用和时间复杂度的总结。
  \item 第三章主要是Min-Min SRLG完全不相交路由算法的设计与实现,在存在陷阱问题的情况下,提出了一种求解Min-Min SRLG完全不相交路由问题的有效算法。为了降低搜索复杂性,提出了一种分而治之的解决方案,将原Min-Min风险共享链路组不相交路由问题划分为多个子问题,该子问题基于从AP路径遇到陷阱问题导出的SRLG冲突链路集,在一个多核CPU平台上使用拓扑上进行了广泛的仿真。
  %\item 第五章介绍了网络虚拟化的相关技术,概述了网络虚拟化环境和网络虚拟化技术特征,接着对虚拟网络嵌入算法分类,总结了不同虚拟网络嵌入算法的优化目的,是否节点/链路映射协调和主要贡献,然后总结了可生存性的虚拟网络嵌入算法故障类型,优化目标和处理机制,最后给出虚拟网络嵌入算法常用的评价度量。
  \item 第四章主要是单物理网络节点故障可生存性虚拟网络嵌入算法的设计与实现,当一个虚拟请求已经嵌入和运行在底层物理网络中时,突然一个底层物理节点随机独立的出现故障,提出的星型分解动态规划节点映射的可生存性虚拟网络嵌入算法可以预先分配备用资源来预防故障失效,进行多种算法指标度量的仿真。
  \item 第五章对现有的工作和研究成果进行了总结,并进一步对其中的某些技术问题探讨了未来研究的方向。
\end{itemize}
