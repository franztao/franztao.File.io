% !Mode:: "TeX:UTF-8"
\section{论文的主要研究内容和组织结构}
\subsection{论文的主要研究内容}
\textbf{缺少SRLG不相交的}

SDN作为新型网络架构,网络虚拟化作为未来网络研究的重要领域,两者的结合具有新的研究意义。在网络虚拟化的过程中,不可避开的一个问题就是虚拟网络嵌入算法的研究,嵌入算法主要考虑的是虚拟网络节点和链路向底层物理网的节点和链路的映射,考虑如何为虚拟网络在底层物理网中找到满足条件的最优路由,并且尽可能地提髙网络资源的利用率。鉴于此,本文做了如下研究:
\begin{enumerate}
  \item \textbf{缺少SRLG不相交的}
  \item 首先,研究SDN虚拟网络技术,对SDN网络虚拟化的环境进行了研究,并 且探讨虚拟网络向底层物理进行嵌入时的不同嵌入算法。

  \item 在虚拟网络向底层物理网络映射时,考虑节点和链路的映射以及路由算法 的设计,在路由算法的设计中,假设节点己经映射并且节点具有功能条件的约束。分别设计算法实现此时提供可生存性的虚拟网络嵌入算法。
\end{enumerate}


本课题将结合位置约束的概念,以虚拟网络请求的接受率、物理网络资源利 用率和故障恢复率等指标作为算法性能评价B标,提出一个。。。。。生存性虚拟网络映射算法。为了实现课题目标,需要进行以下研宄工作:

1.	调研现有的生存性虚拟网络映射算法,以及基于SDN的虚拟网络映射算 法和生存性虚拟网络映射算法,分析不同算法的优缺点,以及存在的共 性问题。

2.	提出一种SDN网络环境下基于资源感知备份策略的生存性虚拟网络映 射算法。。。。。。。。。。。。。。。。。。

3.	设计和搭建仿真平台,对提出的映射算法进行仿真,并实现相关的对比方案。
4.	对实验结果进行分析,评估本文方案和对比方案的性能优劣。


\subsection{论文的组织结构与贡献}
本文依据网络可生存性、不相交路径以及虚拟网络嵌入。等方面的研究现状与发展趋势,并结合自身对以上的探索和研究,将本文分为七章,各章主要内容如下:


\begin{itemize}
  \item 第一章为绪论部分,主要介绍了本课题的研究背景及其意义并且简要地分析了国内外研究现状和发展趋势。
  \item 第二章介绍了网络可生存性相关原理,概述了网络的常见故障和网络故障恢复的两大机制,接着详细介绍了网络故障保护策略。
  \item 第三章介绍了不相交路径的相关技术,概述了不同类型的不相交路径的情形,接着给出不同类型的不相交路径的限制条件,实际应用和时间复杂度总结。
  \item 第四章主要是SRLG不相交路由算法的设计与实现,在存在陷阱问题的情况下,我们提出了一种求解Min-Min SRLG-不相交路由问题的有效算法。为了降低选择对搜索的复杂性,我们提出了一种分而治之的解决方案,将原Min-Min SRLG不相交路由问题划分为多个子问题,该子问题基于从AP路径遇到陷阱问题导出的SRLG冲突链路集。我们的算法利用现有的AP搜索结果和并行执行来实现更快的路径查找。我们在一个多核CPU平台上使用拓扑跟踪进行了广泛的仿真.仿真结果表明,在搜索速度较高的情况下,该算法的性能优于其它算法。
  \item 第五章介绍了网络虚拟化的相关技术,概述了SDN网络虚拟化环境和SDN网络虚拟化技术特征,接着洋细介绍了虚拟网络嵌入算法和可生存性的虚拟网络嵌入算法相关研研究。
  \item 第六章主要是单节点故障可生存性虚拟网络嵌入算法设计与实现,当一个虚拟请求嵌入和运行在底层物理网络中时,突然随机独立的一个底层物理节点出现故障,我提出一个星型分解和动态规划分配的算法,该算法的性能优于其它算法。
  \item 第七章对现有的工作和研究成果进行了总结,并进一步对其中的某些技术问题探讨了改进方案。
\end{itemize}
