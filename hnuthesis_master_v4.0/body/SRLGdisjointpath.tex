\chapter{共享风险链路组不相交路径对算法}

\section{问题描述}
SRLG不相交路径在它们之间没有任何共同的风险资源,也就是说,由于风险而导致的路径失败不会影响其他路径。图\ref{fig:CompositeGraph}(b)显示两条SRLG 不相交路径对,表示为AP 和BP。因为这两条路径没有共同的风险资源,如果AP 失败,BP仍然可以工作。本章主要讨论了两条不相交的路径,即可以描述如下。

\textbf{Min-Min SRLG不相交路径对问题}。给定一个图$G(V,E)$,每条链路$e_i\in \mathbb{E}$ 相关联一个权重$w_{e_i}$,一个源节点$s$和一个终节点$d$,找到一对$s$ 到$d$ 的SRLG不相交路径对(表示为AP和BP),而且要求这两条不相交路径中路径权重较小的那条路径权重最小化,形式化如下:

\begin{equation}
\begin{array}{*{20}{c}}
   {\mathop {minimize}\limits_{AP,BP} } & {\min \left( {{w_{AP}},{w_{BP}}} \right)}  \\
   {subject\ to} & {{r_{AP}} \cap {r_{BP}}{\rm{ = }}\phi }  \\
   {} & {\mathbb{AP} \cap \mathbb{BP}{\rm{ = }}\phi }  \\
\end{array}
\label{eq:problem definition}
\end{equation}

即${w_{AP}}$ 和 ${w_{BP}}$是AP和BP的路径权重,$\mathbb{AP}$ 和 $\mathbb{BP}$分别是路径AP和BP 上的链路集,${r_{AP}}$ 和 ${r_{BP}}$分别是影响路径AP和BP的SRLG 集。


\begin{figure*}[tp]
  \centering
  % Requires \usepackage{graphicx}
  \includegraphics[width=7.2in]{figures/CompositeGraph}
  \caption{SCLS算法实例}
  \label{fig:CompositeGraph}
\end{figure*}

%现如今研究进展
\section{原有算法概述}
共享风险链路组(SRLG)是一组链路共享相同的一个组件,该组件的故障会导致在这个组里所有链路同时发生故障。就路径保护而言,尽管某些链路或者节点不相交路径算法\cite{suurballe1984quick,bhandari1997optimal,li1990complexity,guo2003link,xu2004finding,beshir2011variants,guo2013finding,hu2003diverse} 已经提出来,SRLG不相交路径问题是比较棘手的,而且这些原有研究是限制在一定领域的。比如当每个SRLG 只包含一条链路时,这个SRLG不相交路由问题可以简化为链路不相交路径问题,而通过节点分裂方法(node split method)\cite{ford2015flows}节点不相交路径问题可以转化为链路不相交路径问题。因为SRLG组通常包括的链路超过一条,并且网络中的链路通常可以属于多个SRLG组里,以至于求一对SRLG 不相交路径问题比求一对链路或者节点不相交路径问题要困难得多。

为了解决SRLG不相交路径问题,一种可能的方法是0-1整数线性规划(ILP)\cite{hu2003diverse},通过分支限界法(branch-and-bound)来搜索选择最优的主路径和备份路径。该方法时间复杂度高,不适用于大型网络。为了降低算法的复杂度,基于APF的启发式算法\cite{oki2002disjoint,li2002fiber,eppstein1998finding}能够求Min-Min SRLG 不相交路径问题的近似最优解。首先使用Dijkstra算法(或任何其他最短路径算法)求出主路径,求主路径时不考虑其相应的备用路径情况,在删除AP沿线的链路并且与AP共风险的节点和链路后,再利用最短路算法求的备用路径。

然而,使用APF启发式算法的有一个主要缺陷,一旦求得路径AP后也可能无法找到相对的SRLG不相交路径BP,即使网络中确实存在一对不相交路径。这就是所谓的“陷阱”问题,在节\ref{sec:trapproblem} 将详细介绍陷阱问题产生的原因,即使稠密网络中\cite{laborczi2001solving}这也是可能发生,在一个稀疏连接的网络中当然是不能被忽略。陷阱问题分为两种:不可避免的陷阱和可避免的陷阱。不可避免的陷阱是受拓扑约束的,任何算法都无法解决。如果网络不是2-边连通度的,则没有算法可以保证在拓扑中存在两个SRLG不相交路径。另一方面,当两个节点之间存在SRLG不相交路径对,但由于路由算法的缺陷而找不到时,就会出现一个可避免的陷阱,在本章中只考虑了可避免的陷阱。

对简单的APF算法扩展,提出了KSP(K-最短路径)算法来处理节点/链路不相交路径的陷阱问题。虽然它是处理陷阱问题最有效的算法之一,但它在大型网络中的性能受到影响,因为KSP 可能会涉及多路径搜索测试(K测试),直到它找到不相交路径。当前候选的路径AP遇到陷阱问题后,仅根据路径长度选择下一个要测试的候选AP,而不考虑当前候选AP 的那条链路(或那些链路)导致查找不相交路径BP失败。因此,为了找到一对不相交路径对,需要对大量的路径进行测试,这就引入了KSP算法中与K 相关的时间复杂度。对于遇到陷阱问题的AP,我们应用从AP 路径导出的SRLG冲突链路集来指导将来的AP路径测试。这在很大程度上有助于减少寻找替代路径的时间复杂度。

其它SRLG不相交路径算法\cite{rostami2012msdp,rostami2007cose,datta2008graph,xu2003new,todimala2004imsh},搜索最大SRLG不相交路径对,并且路径间共享最小数目的公共链路。由于AP 和BP 可能具有相同的风险资源组,通过这种方法找到的解决方案是不可靠的。我们的算法目标是寻找完全SRLG不相交路径。Xu\cite{xu2003trap}试图找到完全SRLG 不相交路径。但他的算法减少了问题的搜索空间,加快了路径搜索的速度。然而,它可能会以较大的代价返回路径,因为在削减后的搜索空间可能会失去最优解。相反,为了大大加快搜索过程,我们利用SRLG冲突链路集将原问题划分为多个子问题,这些子问题可以并行执行。因此,我们的算法可以运行得更快,返回主路径成本非常低。

Datta\cite{datta2008graph}提出方法是将SRLG不相交路径问题转化为链路不相交路径问题,然后利用链路不相交路径算法来解决。然而,只有特殊的SRLG 星型模式可以转换为链路不相交,这样就限制了该算法的广泛应用。当AP遇到陷阱问题时,CoSE\cite{rostami2007cose}算法试图找到一个SRLG集合,任何AP 路径包含了这个SRLG集合里的所有SRLG,则必定找不到任何的与其对应的SRLG不相交路径BP。CoSE 首先通过多轮搜索查找多个AP共享的SRLG,并且组成一个SRLG集合,然后根据SRLG 集合来划分原始问题以搜索SRLG不相交路径对。而不使用SRLG 中链路之间共享风险的特性,CoSE方法的这种穷尽搜索需要非常高的计算开销。



%\section{整数规划形式化}
\section{复杂度规模}
\begin{theorem}
\label{le:lemma1}
    Min-Min SRLG-不相交路径对问题是 NP-complete问题.
\end{theorem}
\begin{proof}
根据\cite{bhatia2006finding},Min-Min链路不相交路径对问题是NP-complete 的。Min-Min链路不相交路径问题是Min-Min SRLG不相交路径问题的子问题。设
Min-Min SRLG不相交路径问题的复杂性为C(A),则NP-complete$\leq$C(A).

为了求Min-Min SRLG不相交路径问题的时间复杂度,我们首先假设了一个问题B(问题B的复杂度表示为C(B)),当找到两条的SRLG不相交路径并且路径较小的路径其权重小于或等于M(M是大于零的整数)。Min-Min SRLG不相交路径问题A 与问题B等价,我们知道M必须大于零并且小于$\sum\limits_{e_i\in \mathbb{E}}w_{e_i}$。例如,我们假设0≤M≤10 和m=6 是最优解,通过经典的二分法(binary search method)其时间复杂度为O(log(N)),如图\ref{fig:binarySearch}所示,通过二分法我们得到了两条不相交的路径,较小的路径其权重为m,因此问题A与 问题B等价。
\begin{figure}[htbp]
  \centering
  % Requires \usepackage{graphicx}
  \includegraphics[width=4.0in]{figures/binarySearch}
  \caption{二进制搜索法求最优解实例}
  \label{fig:binarySearch}
\end{figure}
假设程序X在输入问题B时,如果问题B没有解,则程序Y立即停止,否则程序B继续执行并获得问题B的解,因此问题B 可以归结为NP-hard问题。因此C(B)$\leq$NP-hard。

此外,给定任意两条路径,很容易在多项式时间内判别这两条路径是否为SRLG 不相交路径,较小的路径其权重小于或等于M,从而使得C(B)$\leq$NP-complete。 当B的复杂度等于A时,我们有C(A)=C(B)$\leq$NP-complete。因此,A=NP-complete。
\end{proof}
\section{陷阱问题}
\label{sec:trapproblem}
基于APF的启发式算法可能会陷入“陷阱”问题。也就是说,当一个AP被确定时,即使网络中确实存在一对不相交路径对,它也可能无法找到SRLG不相交的BP路径。图ref{fig:CompositeGraph}.(c),(d)说明了陷阱问题,虚线表示一条AP 路径,其链路集为$\mathbb{AP}=\{e_1,e_2,e_3$ $,e_4,e_5,e_6,e_7,e_8\}$,在删除AP上的链路和与AP 共享风险的链路后,图\ref{fig:CompositeGraph}.(d) 所示的不存在从s到d的路径,因此找不到BP。

虽然KSP算法被认为是解决陷阱问题的有效算法,但它可能面临着效率低下的问题。如图\ref{fig:KSPproblem}所示假设$e_1, e_2, e_3, e_4$的链路权重比其他链路大得多。此外,在$e_1, e_2, e_3, e_4$中,$e_1$ 和$e_2$的链路权重远小于$e_3$,$e_4$。然后,在KSP算法多次找从s到d的K 短路时,总是包含$e_1,e_2$ (虚线表示)。则最短AP 总会遇到陷阱问题,因为$e_1$和$e_4$ 具有相同的风险,因此无法找到BP。为了避免陷阱问题,必须将K设为一个大值,这给KSP带来了很高的时间复杂度。
\begin{figure}[htbp]
\centering
% Requires \usepackage{graphicx}
\includegraphics[width=4.0in]{figures/KSPproblem}
  \caption{演示KSP算法效率低下的实例}
  \label{fig:KSPproblem}
\end{figure}


\section{分而治之的快速SRLG不相交路径对算法}
当一个陷阱问题发生,并且对于给定的AP没有SRLG不相交路径BP时,AP中可能存在一个子链路集,这样任何通过这个链路集里所有的这些“问题”链路的AP都不能找到一条相对应SRLG不相交BP路径,我们称之为\textbf{SRLG冲突链路集}。与KSP 不同,当最短AP路径遇到陷阱问题时,我们将通过两个主要步骤来解决这个问题。如图\ref{fig:KSPproblem} 所示的例子中,我们将首先找到图\ref{fig:KSPproblem} 中的SRLG冲突链路集合,然后应用分而治之算法将原问题划分为两个子问题$\mathcal{P}(\emptyset,\{e_1\})$和 $\mathcal{P}(\{e_1\},\{e_2\})$。 这两个子问题可以在多核CPU平台上并行执行,快速得到SRLG不相交路径对。
\subsection{分而治之}
在得到SRLG冲突链路集后,设计了一种分而治之的算法,将原Min-Min SRLG不相交路径问题划分为多个子问题,并行执行,加快求SRLG不相交路径对的过程。

为了便于问题划分,我们首先定义了两个不相交的链路集合$\mathbb{I}$和$\mathbb{O}$,其中称$\mathbb{I}$为包含集集合,$\mathbb{O}$称为排除集集合。由$\mathcal{P}({\mathbb{I},\mathbb{O}})$表示的Min-Min SRLG不相交问题,用于寻找一对AP和BP,其中AP是所有可能的AP中最短的,其中路径$AP$必须经过$\mathbb{I}$集合里的所有链路和不经过$\mathbb{O}$集合里的所有链路。

最初,让$\mathbb{I}=\emptyset$ ${\mathbb{O}}=\emptyset$,原来的Min-Min SRLG不相交路径对问题可以用$\mathcal{P}(\emptyset,\emptyset)$ 表示。给定SRLG冲突链路集$\mathbb{T}=\{{e_1},{e_2}, \cdots ,{e_{\left| \mathbb{T} \right|}}\}$,原问题$\mathcal{P}(\emptyset,\emptyset)$可按以下步骤划分。

\begin{enumerate}
  \item 首先,$\mathcal{P}(\emptyset,\emptyset)$能被划分成两个子问题$\mathcal{P}(\emptyset,\{e_1\})$ 和 $\mathcal{P}(\{e_1\},\emptyset)$。
  \item 类似,$\mathcal{P}(\emptyset,\{e_1\})$能被划分成两个子问题 $\mathcal{P}(\{e_1,e_2\},\emptyset)$ 和 $\mathcal{P}(\{e_1\},\{e_2\})$。
  \item 这个划分步骤持续直到步骤$|\mathbb{T}|$,问题$\mathcal{P}(\{e_1,e_2,\cdots ,{e_{\left| \mathbb{T} \right|-1}}\},\emptyset)$ 进一步的拆分成两个子问题$\mathcal{P}(\{e_1,e_2,\cdots ,{e_{\left| \mathbb{T} \right|-1}}, {e_{\left| \mathbb{T} \right|}}\},\emptyset)$ 和 $\mathcal{P}(\{e_1,e_2,\cdots ,{e_{\left| \mathbb{T} \right|-1}}\},{e_{\left| \mathbb{T} \right|}})$。注意到,子问题$\mathcal{P}(\{e_1,e_2,\cdots ,{e_{\left| \mathbb{T} \right|-1}}, {e_{\left| \mathbb{T} \right|}}\},\emptyset)$是无解的。
\end{enumerate}



除了子问题$\mathcal{P}(\{e_1,e_2,\cdots ,{e_{\left| \mathbb{T} \right|}}\},\emptyset)$外,我们将试图求其它每个子问题的最优解。然后选择最好的路径对(即主路径最短的路径对),作为原问题$\mathcal{P}(\emptyset,\emptyset)$的最终(最优)解。如果这些子问题都没有解,则我们可以得出原问题没有任何解,因为我们子问题包括了所有的可能的不相交路径对。

就时间复杂性而言,解决子问题所需的时间比原来的问题应该花费的更少。因为一条链路(来自集合$\mathbb{T}$)将在计算AP的路径时被去除,这也确保了不同的AP 路径将被测试且是否存在一个SRLG不相交BP 路径。

当遇到陷阱问题时,我们的解决方案将划分原来的问题,并测试每个子问题以寻找到最终的解。在我们的分而治之方法中,子问题是由SRLG冲突链路集而得,而这个SRLG冲突链路集却是当AP路径遇到陷阱问题生成的。与现有的算法相比较,该算法在不考虑现有结果和问题的情况下,可以在很大程度上降低算法的计算量。对于图\ref{fig:DividedConquer}中的例子所示,SRLG冲突链路集是$\mathbb{T}=\{e_2,e_5,e_6\}$。拆分过程过程如图\ref{fig:DividedConquer} 所示。根据SRLG冲突链路集,我们应该测试总共3个子问题${{\mathcal{P}}(\{ e_2,e_5\} ,\{ e_6\} )}$, ${{\mathcal{P}}(\{ e_2\} ,\{ e_5\} )}$ 和 ${{\mathcal{P}}(\emptyset ,\{ e_2\} )}$,其中选择AP路径权重最低的最优子问题作为原问题$\mathcal{P}(\emptyset,\emptyset)$最终的(最优)解。

注意,我们不需要解决子问题${{\mathcal{P}}(\{ e_2,e_5\}, \emptyset)}$ 和 ${{\mathcal{P}}(\{ e_2\},\emptyset )}$,因为它们的解已经包含在其它的子问题中。第一个解空间由两个子问题${{\mathcal P}(\{ e_2,e_5,e_6\} ,\emptyset )}$ 和 ${{\mathcal P}(\{ e_2,e_5\} ,\{ e_6\} )}$ 组成。由于SRLG冲突链路集为$\mathbb{T}=\{e_2,e_5,e_6\}$,显然,子问题${{\mathcal P}(\{ e_2,e_5,e_6\} ,\emptyset )}$是没有解的。因此,${{\mathcal P}(\{ e_2,e_5\} ,\{ e_6\} )}$的解空间等于${{\mathcal{P}}(\{ e_2, e_5\}, \emptyset)}$的解空间。同样,${{\mathcal{P}}(\{ e_2\},\emptyset )}$的解空间包括${{\mathcal{P}}(\{ e_2\} \{ e_5\}, \emptyset)}$  和 ${{\mathcal{P}}(\{ e_2\} ,\{ e_5\} )}$的解空间。
\begin{figure}[htbp]
\large{
\begin{equation*}
{\mathcal P}(\emptyset ,\emptyset )\left\{ {\begin{array}{*{20}{l}}
{{\mathcal P}(\{ e_2\} ,\emptyset )\left\{ {\begin{array}{*{20}{l}}
{{\mathcal P}(\{ e_2,e_5\} ,\emptyset )\left\{ {\begin{array}{*{20}{l}}
{{\mathcal P}(\{ e_2,e_5,e_6\} ,\emptyset )}\\
{\boxed{{\mathcal P}(\{ e_2,e_5\} ,\{ e_6\} )}}
\end{array}} \right.}\\
{\boxed{{\mathcal P}(\{ e_2\} ,\{ e_5\} )}}
\end{array}} \right.}\\
{\boxed{{\mathcal P}(\emptyset ,\{ e_2\} )}}
\end{array}} \right.
\end{equation*}
}
\caption{分而治之的解决方案}
\label{fig:DividedConquer}
\end{figure}




\subsection{SRLG冲突链路集合}
在本节中,我将描述了如何找到一个SRLG冲突链路集合,当在网络$G$中给定一个AP 路径并且没有SRLG 不相交的BP路径。
\subsubsection{通过新奇的边容量设置准则构造一个新图$G^*$}
如\ref{subsubsec:maxFlow}节介绍的,如果在图G中去除所有在边割集$\mathbb{\mathbb{L}}_{\Phi}$ 的所有边,则$|f| = 0$。也就是说,不存在任何流能从$s$到$d$。在本文中,我们试图基于割集的概念找到SRLG冲突链路集合。如果AP从s到d流通,经过的链路与割集$\mathbb{\mathbb{L}}_{\Phi}$共享风险,则找不到任何与其对应的SRLG不相交路径BP,因为没有一条在割集中的链路可以被BP 选择。

基于割集基础为了找到SRLG 冲突链路集,我们构造了一个新图$G^*$ ,如下所示。
\begin{enumerate}
  \item $G^*$与$G$的节点和链路拓扑关系一样。
  \item 跟每条链路$e_i$相关的链路权重$w_{e_i}$是跟其相对应图$G$中边的权重一样的。
  \item 我们使用公式\ref{eq:capacity principle}的准则设置每条边$e_i \in \mathbb{E}$相关的容量$c_{e_i}$。
\end{enumerate}
\begin{equation}
c_{e_i} = \left\{ {\begin{array}{*{20}{c}}
   1 & {e_i{\rm{ }} \in {\rm{ \mathbb{AP}}}}  \\
   {\left| \mathbb{AP} \right|+1} & {e_i{\rm{ }} \in {\rm{ \mathbb{E}}}{{\rm{\mathbb{R}}}}}  \\
   {\left| {{\rm\mathbb{AP}}} \right| + \left( {\left| {{\rm\mathbb{AP}}} \right| + 1} \right)\times \left| {{\rm{\mathbb{E}}}{{\rm{\mathbb{R}}}}} \right| + 1} & {otherwise}  \\
\end{array}} \right.
\label{eq:capacity principle}
\end{equation}
$\mathbb{AP}$指在图$G$中较小权重路径$AP$上所有链路的集合,和$\mathbb{\mathbb{ER}}$指不属于路径$AP$上的边但是与路径$AP$上的边共享风险的链路集合。

如图\ref{fig:CompositeGraph}(c)所示,路径$AP$的边集合$\mathbb{AP}=\{e_1,e_2,e_3,e_4$
$,e_5,e_6,e_7,e_8\}$, $\mathbb{\mathbb{ER}}=\{e_9,e_{11},e_{17},e_{13},e_{19}\}$。$|\mathbb{AP}|=8$, $|\mathbb{\mathbb{ER}}|=5$, $|\mathbb{AP}|+1=9$ 和 ${\left| {{\rm{\mathbb{AP}}}} \right| + \left( {\left| {{\rm{\mathbb{AP}}}} \right| + 1} \right)\times \left| {{\rm{\mathbb{E}}}{{\rm{\mathbb{R}}}}} \right| + 1}=54$。我们产生一个新图$G^*$如图\ref{fig:FlowStarGraph}所示,在图$G^*$中边的容量是根据公式(\ref{eq:capacity principle})所设置。

$r_P$表示影响路径$P$上的风险集合,即$r_P=\{r\in \mathbb{R}$: 路径 $P$ 包含的链路在 $\mathbb{R}_r$ 中$\}$。 如图\ref{fig:CompositeGraph}(c) 所示,在路径$AP$ 上的边集$\mathbb{AP}=\{e_1,e_2,e_3,e_4,e_5,e_6,e_7,e_8\}$,并且$e_1\in \mathbb{R}_{r_1}$, $e_2\in \mathbb{R}_{r_2}$, $e_2\in \mathbb{R}_{r_3}$, $e_3\in \mathbb{R}_{r_2}$, $e_4\in \mathbb{R}_{r_3}$, $e_5\in \mathbb{R}_{r_4}$,路径$AP$ 的风险集合是${r}_{{AP}}=\{r_1, r_2, r_3, r_4\}$。$\mathbb{\mathbb{ER}}$ 代表不属于$AP$上的链路但是与$AP$共享相同的风险集合的链路。如图\ref{fig:CompositeGraph}(c) 所示,$\mathbb{\mathbb{ER}}=\{e_9,e_{11},e_{17},e_{13},e_{19}\}$。

\begin{figure}[tp]
  \centering
  % Requires \usepackage{graphicx}
  \includegraphics[width=4.5in]{figures/FlowStarGraph}
  \caption{图$G^*$实例}\label{fig:FlowStarGraph}
\end{figure}



\subsubsection{最小割最大网络流定理}
\label{subsubsec:maxFlow}
设$G=(\mathbb{\mathbb{V}},\mathbb{\mathbb{E}})$是一个网络(其中$\mathbb{\mathbb{V}}$是$|\mathbb{\mathbb{V}}|$个节点的集合,$\mathbb{\mathbb{E}}$是$|\mathbb{\mathbb{E}}|$条链路的集合),其中$s\in \mathbb{V}$和$d\in \mathbb{V}$分别指源节点和终节点。链路$e_i$ 的\textbf{容量}表示该条链路的最大流量。链路的流$f_{e_i}$应该满足以下两个限制:
\begin{enumerate}
  \item 容量限制: $\forall e_i\in \mathbb{\mathbb{E}}$: $f_{e_i}\leq c_{e_i}$.
  \item 流量守恒: $\forall u\in \mathbb{\mathbb{V}}-\{s,d\}$: $\sum\limits_{v\in \mathbb{V}}f_{(v,u)}=\sum\limits_{v\in \mathbb{V}}f_{(u,v)}$,  $(v,u)$ 和 $(u,v)$ 代表链路 $e(v,u)$ 和 $e(u,v)$.
\end{enumerate}

流的值定义为$|f|=\sum\limits_{v\in \mathbb{V}}f_{(s,v)}$,其中s是源节点。它表示从s节点到d 节点的流量。\textbf{最大流量问题}:尽可能的求从s 节点到d 节点的最大流量值$|f|$。

一个 s-d 割${\Phi}=(\mathbb{S},\mathbb{D})$ 是节点$\mathbb{V}$的划分且$s \in \mathbb{S}$ 和 $d \in \mathbb{D}$,$\Phi$ 的割集合$\mathbb{\mathbb{L}}_{\Phi}$ 是一个包含边的集合。
\begin{equation}
\mathbb{\mathbb{L}}_{\Phi}=\{(u,v)\in \mathbb{E}: u \in \mathbb{S}, v \in \mathbb{D}\}.
\end{equation}

如果在割集合$\mathbb{\mathbb{L}}_{\Phi}$中的边被去除,那么在原图中的流值$|f| = 0$。即没有流能从s 节点到d 节点。一个 s-d 割${\Phi}=(\mathbb{S},\mathbb{D})$ 的容量被定义成$c(\Phi)=\sum\limits_{e_i\in \mathbb{\mathbb{L}}_{\Phi}}c_{e_i}$。\textbf{最小s-d 割 $\Phi$ 问题},最小化$c(\Phi)$即决定点集$\mathbb{S}$ 和 $\mathbb{D}$使得s-d 割${\Phi}=(\mathbb{S},\mathbb{D})$的($c(\Phi)$) 最小化 。

\textbf{最小割最大流定理}:一个s-d流的最大值等于s-d割的最小割。如图\ref{fig:FlowNetwork}所示,在图$G$ 中的流量$|f|=f_{(s,v_1)}+f_{(s,v_2)}$。 这个割$\Phi(\mathbb{S},\mathbb{D})$ 是 $\mathbb{S}=\{s,v_1,v_2,v_4\}$ 和$\mathbb{D}=\{v_3,d\}$,它是最小的割其容量为$c(\Phi)=c_{(v_1,v_3)}+c_{(v_4,v_3)}+c_{(v_4,d)}=12+7+4=23$. 显然, $|f|=c(\Phi)$, 即s-d 最大流等于所有s-d割中最小的容量。
\begin{figure}[htbp]
  \centering
  % Requires \usepackage{graphicx}
  \includegraphics[width=4.0in]{figures/FlowNetwork}\\
  \caption{最大流最小割定理实例}
  \label{fig:FlowNetwork}
\end{figure}

\subsubsection{新图$G^*$中最小割集性质}
最大流最小割定理指出,在网络中,从源节点$s$到目的节点$d$的最大流值等于最小边割集的总容量,即最小的链路总容量,如果删除最小边割集上的边,将断开目的节点$d$与源节点$s$的连通。 我们的算法基于图的最小割集的链路集来求得最小SRLG 冲突链路集。我们将首先展示我们重建新图$G^*$的一些优良特性为求得最小SRLG冲突链路集。

\begin{lemma}
\label{le:lemma1}
    在图$G^*$中任何从$s$到$d$的路径必须经过在$\mathbb{AP}$ 或者 $\mathbb{\mathbb{ER}}$中边集合的一条边。
    %note:draw a picture to describe the proof
\end{lemma}
\begin{proof}
用矛盾法证明这条引理。假设对于某条路径AP,有另一条路径从$s$到$d$ 在图$G^*$中不与AP共享风险,即这条路径不通过$\mathbb{AP}$ 或者 $\mathbb{\mathbb{ER}}$的任何一条链路。可以很容易得出这样的结论,这条路径是与路径AP对应的SRLG不相交路径BP。这与AP没有对应的SRLG不相交路径的说法相矛盾。
\end{proof}

\begin{lemma}
\label{le:lemma2}
    图$G^*$的任何一条最大流的流值最多为$|\mathbb{AP}|+(|\mathbb{AP}|+1)\times|\mathbb{\mathbb{ER}}|$。
\end{lemma}
\begin{proof}
假设$G^*$的最大流$f$为$|f|=k$。在$G^*$中,f可以被划分为$k$个从s到d的1 单元流。根据引理.\ref{le:lemma1},这些1-单位流中的每一个都必须通过通过$\mathbb{AP}$ 或者 $\mathbb{\mathbb{ER}}$的至少一条链路。请注意,$\mathbb{AP}$ 或者 $\mathbb{\mathbb{ER}}$ 中链路的容量分别为1或者$|\mathbb{AP}|$+1。根据$\mathbb{AP}$ 和 $\mathbb{\mathbb{ER}}$中链路的容量设置,因为$\mathbb{AP}$中的链路只能承载1-单位流量,而在ER中一条链路在$\mathbb{ER}$中能最多承载$|\mathbb{AP}|$+1单位流量。因此,最多只能有$|\mathbb{AP}|+ (|\mathbb{AP}|+1)\times|\mathbb{\mathbb{ER}}|$ 个从s到d 的单位流量。
\end{proof}

\begin{lemma}
\label{le:lemma3}
    图$G^*$最小割$\Phi$的边割集$\mathbb{L}_{\Phi}$上所有链路都在$\mathbb{AP}$ 或者$\mathbb{\mathbb{ER}}$ 中。
\end{lemma}

\begin{proof}
根据最大流最小割定理,$c(\Phi)$表示的最小切割$\Phi$的的容量,其应该等于最大流量值,根据引理 \ref{le:lemma2} 最大流量最多为$|\mathbb{AP}|+ |\mathbb{ER}|\times (|\mathbb{AP}|+1)$。 根据容量设定原则如公式\ref{eq:capacity principle}所示,一条链路既不在$\mathbb{AP}$ 也不在 $\mathbb{ER}$ ($e_i \notin \mathbb{AP}$ 和 $e_i \notin \mathbb{ER}$), 则这条链路的容量为 $c_{e_i} = \left| {{\rm\mathbb{AP}}} \right| + \left( {\left| {{\rm\mathbb{AP}}} \right| + 1} \right)\times \left| {{\rm{\mathbb{E}}}{{\rm{\mathbb{R}}}}} \right| + 1$, 这条链路是大于$|\mathbb{AP}|+(|\mathbb{AP}|+1)\times |\mathbb{ER}|$。 因此,这条链路不可能在$\mathbb{L}_{\Phi}$ 中。因此,割集$\mathbb{L}_{\Phi}$中的所有链路都必须属于边集合$\mathbb{AP}$ 或者 $\mathbb{ER}$。
\end{proof}


\begin{theorem}
    如果在图$G$中一个单元流阻塞了全部边割集$\mathbb{L}_{\Phi}$的所有边,则在原图中不会存在任何流经过这个图的割$\Phi$。
\label{th:block flow}
\end{theorem}


\begin{proof}
    如果在图$G$中一个单元流阻塞了全部边割集$\mathbb{L}_{\Phi}$的所有边,则在原图中没有流能使用边割集$\mathbb{L}_{\Phi}$ 里的边和没有任何流能经过这个图的割$\Phi$。
\end{proof}
定理\ref{th:block flow}提供了找到SRLG冲突链路集的可能性。即当AP路径遇到陷阱问题时,我们找到$\mathbb{AP}$ 路径边集的子集能够阻塞原有在边割集$\mathbb{L}_{\Phi}$的所有边,所以这个子集合即为SRLG冲突链路集。当一条路径包含SRLG冲突链路集里的所有边,则没有任何流能经过割集$\Phi$,因此没有与其对应的SRLG不相交路径BP。

\subsubsection{SRLG冲突链路集的集合覆盖问题}
\label{subsec:Set cover problem for SRLG Conflicting Link Set}
虽然AP路径上的所有链路一起也可以构成SRLG冲突链路集,我们感兴趣的是尽量规模较小的SRLG冲突链路集,SRLG 冲突链路集的大小确定相互间互斥的子问题规模。

根据定理\ref{th:block flow},最小SRLG冲突链路集问题可以描述为:查找AP 链路集上的最小链路子集,这些链路可以阻塞边割集$\mathbb{L}_{\Phi}$。

对于任何链路$e_i$,$\mathbb{SR}_{e_i}$表示与链路$e_i$共享风险的链路集。显然,$\mathbb{SR}_{e_i}$ 包括$e_i$ 本身和所有包含链路$e_i$风险贡献链路组SRLG集合的所有边。例如,如图\ref{fig:MinCutStarGraph}所示,$e_i$ 是在两个SRLG中$\mathbb{R}_{r_2}=\{e_2,e_3,e_{19}\}$, $\mathbb{R}_{r_3}=\{e_2,e_4,e_{11},e_{17}\}$,因此, $\mathbb{SR}_{e_2}=\{e_2,e_3,e_{19},e_4,e_{11},e_{17}\}$。

对每条在AP上的路径$e_i$,我们定义每条边的cut-block-link集合为${\mathbb{B}_{{e_i}}} = \mathbb{SR}_{{e_i}} \cap \mathbb{L}_{\Phi}$,这个集合是边割集$\mathbb{L}_{\Phi}$的子集,能通过$e_i$而堵塞这个集合的所有边。

因此,最小SRLG冲突链路集问题可以定义为一个集合覆盖问题:给定$\mathbb{AP}$(路径AP上的链路集合)、边割集$\mathbb{L}_{\Phi}$和cut-block-link集合${\mathbb{B}_{{e_1}}},{\mathbb{B}_{{e_2}}}, \cdots ,{\mathbb{B}_{{e_{|\mathbb{AP}|}}}}$。 我们求出最小cut-block-link集合集,其交集是边割集$\mathbb{L}_{\Phi}$,即最小规模的$\mathbb{T} \subseteq \{e_i| e_i\in \mathbb{AP}\}$ 以至于 ${ \cup_{e_i \in \mathbb{T}}}{\mathbb{B}_{e_i}} = \mathbb{L}_{\Phi}$。


\begin{figure*}[htbp]
  \centering
  % Requires \usepackage{graphicx}
  \includegraphics[width=4.5in]{figures/MinCutStarGraph}
  \caption{图$G^*$的最小割实例}\label{fig:MinCutStarGraph}
  \label{fig:MinCutStarGraph}
\end{figure*}

集合覆盖问题通常是一个NP-hard问题。它的复杂性取决于元素的大小(表示($N$)。 在我们的最小SRLG 冲突链路集问题中,$n=|\mathbb{L}_{\Phi}|$,即割集$\mathbb{L}_{\Phi}$的边数。因为本文的重点不是改进集合覆盖问题算法,我们应用\cite{chvatal1979greedy}中提出的算法。在复杂度$O(log(|\mathbb{L}_{\Phi}|))$的情况下求出最小的SRLG 冲突链路集,通常即使是在大规模的网络,最短的路径作为AP都没有大跳数$n=|\mathbb{L}_{\Phi}|$。因此,用集合覆盖问题来求最小值SRLG冲突链路集代价不是很大。

为了说明如何找到最小SRLG冲突链路集,如图\ref{fig:MinCutStarGraph} 所展示了一个例子,求最小$\Phi(\mathbb{S},\mathbb{D})$, $\mathbb{S}=\{s, 1, 2, 3, 4, 5, 9\}$和$\mathbb{D}=\{d, 6, 7, 8\}$,边割集$\mathbb{L}_{\Phi}=\{e_{11},e_{13},e_{19},e_{6}\}$。对于AP路径上的所有链路,cut-block-link集合是:$\mathbb{B}_{e_1}=\emptyset$, $\mathbb{B}_{e_2}=\{e_{11},e_{19}\}$, $\mathbb{B}_{e_3}=\{e_{19}\}$, $\mathbb{B}_{e_4}=\{e_{11}\}$, $\mathbb{B}_{e_5}=\{e_{13}\}$, $\mathbb{B}_{e_6}=\{e_6\}$, $\mathbb{B}_{e_7}=\emptyset$ and $\mathbb{B}_{e_8}=\emptyset$。 为了覆盖$\mathbb{L}_{\Phi}$,最少的cut-block-link集合是$\mathbb{B}_{e_2}=\{e_{11},e_{19}\}$,$\mathbb{B}_{e_5}=\{e_{13}\}$,$\mathbb{B}_{e_6}=\{e_6\}$。 因此,最小SRLG冲突链路集是$\mathbb{T}=\{e_2, e_5, e_6 \}$。

如图\ref{fig:MinCutStarGraph}所示的例子中,虽然$|\mathbb{L}_{\Phi}|=4$,但是最小SRLG冲突链路集$|\mathbb{T}|=|\{e_2, e_5, e_6 \}|=3$是小于$|\mathbb{L}_{\Phi}|=4$。这是因为$e_2$属于$\mathbb{R}_{r_2}$ 和$\mathbb{R}_{r_3}$,并能阻塞在割集$\mathbb{L}_{\Phi}$中的两条链路$e_{11}$ 和$e_{19}$


根据SRLG的拓扑类型\cite{datta2008graph},SRLG的拓扑类型:星型类型和非星型类型。对星型类型SRLG,所有链路都从同一个节点开始或结束在同一个节点。例如,如图\ref{fig:MinCutStarGraph}所示,$e_1$和$e_9$来自相同的节点$s$,$\mathbb{R}_{r_1}$是星型SRLG。对非星型类型,并不是SRLG 中的所有链路都是从相同节点开始或结束于同一节点。如图\ref{fig:MinCutStarGraph} 所示,$\mathbb{R}_{r_2}$, $\mathbb{R}_{r_3}$, $\mathbb{R}_{r_4}$ 和 $\mathbb{R}_{r_5}$是非星型类型。而且,即使在图\ref{fig:MinCutStarGraph} 所示包括星型SRLG 和无星型SRLG,我们的算法高效且有效地通过解决集合覆盖问题来解决冲突集。


因此,与现有的一些研究不同的是,\cite{datta2008graph}只能处理单个SRLG 类型,这样的简单场景中一条链路只属于一个SRLG,我们的算法能更有效的处理多种情形。在更一般的情况下,链路可以属于一个或多个SRLG具有更多不同的SRLG 类型。

\subsection{算法时间复杂度}
\label{subsec:Complexity analysis}
当AP遇到陷阱问题时为了找到SRLG不相交路径对,我的算法首先计算出SRLG冲突链路集,然后通过把原问题划分成T个子问题来求解原问题。如\ref{subsec:Set cover problem for SRLG Conflicting Link Set}节所述边割集$\mathbb{L}_{\Phi}$ 的边数规模通常不是很大,因此,本算法寻找SRLG冲突链路集不会带来太多的时间成本。因此,我们关注的是路径查找过程的计算开销。

一般来说,对于一个有$|\mathbb{E}|$条链路和$|\mathbb{V}|$个节点的网络,求最小权重路径问题的时间复杂性是$(|\mathbb{E}|+|\mathbb{V}|)\times log(|\mathbb{V}|)$。为了解决陷阱问题,我们的路径查找问题与最初的最小权重路径问题有点不同。我们在路径查找的过程中引入了一些约束条件。例如,查找AP路径必须通过必过链路集$\mathbb{I}$和必不过链路集$\mathbb{O}$。由于这些链路集通常并不大,这些约束在成本计算上几乎没有差别。因为不同的子问题有不同的链路集,为了使描述简单明了,我们仍然使用$(|\mathbb{E}|+|\mathbb{V}|)\times log(|\mathbb{V}|)$ 作为一次路径搜索时间复杂度。而我们算法将原问题分成$|\mathbb{T}|$个子问题,算法复杂度为$|\mathbb{T}|\times(|\mathbb{E}|+|\mathbb{V}|)\times log(|\mathbb{V}|)$。


对于不同算法复杂性的比较,我们也展示了在CoSE\cite{rostami2007cose}和KSP\cite{eppstein1998finding}在路径查找过程中的复杂性。

CoSE试图找到一个冲突的SRLG集合,而不是一个冲突链路集。但是他们寻找冲突的SRLG集的方法是穷尽的查找而且成本很高。在这我们主要分析研究路径查找过程时间成本。由于我们的SRLG冲突链路集是由最小割和集合覆盖问题导出的,$|\mathbb{T}|$是最小的SRLG冲突链路集规模。因此,在CoSE中的冲突SRLG集至少是$|\mathbb{T}|$,并且我们表示SRLG集合为$\left\{ {SRL{G_1},SRL{G_2}, \cdots ,SRL{G_{|\mathbb{T}|}}} \right\}$,由于每个SRLG路径包含多条链路,因此CoSE的子问题应该比我们的要大得多。在AP路径上的一个SRLG 的必过链路集合和不过链路集会产生$|SRLG|$个子问题。所以一个SRLG集合$\left\{ {SRL{G_1},SRL{G_2}, \cdots ,SRL{G_{|\mathbb{T}|}}} \right\}$ 将引入$\prod\limits_{i = 1}^{_{\left| T \right|}} {\left| {SRL{G_i}} \right|}$个子问题。因此CoSE的复杂性是$\prod\limits_{i = 1}^{_{|\mathbb{T}|}} {\left| {SRL{G_i}} \right|}\times (|\mathbb{E}|+|\mathbb{V}|)\times log(|\mathbb{V}|)$,这是比我们的算法那复杂度大得多的。

对于KSP算法\cite{eppstein1998finding},路径查找复杂度为$K\times ((|\mathbb{E}|+|\mathbb{V}|)\times log(|\mathbb{V}|))$,其中K是在发现SRLG不相交路径对之前应该测试的路径次数。然而,由于KSP没有从前面的路径搜索过程利用前面的信息,在最坏的情况下,KSP可能尝试从源节点$s$到目的节点$d$的所有路径。因此,最糟糕的K值是$2^{|\mathbb{E}|}$,这会带来很大的计算成本。

