% !Mode:: "TeX:UTF-8"

\chapter{总结和展望}
\section{本文工作总结}
本文提出了一种在存在陷阱问题情况下求解Min-Min SRLG不相交路由问题的有效算法。为了降低选择对搜索的复杂性,我们提出了一种分而治之的解决方案,将原Min-Min SRLG不相交路由问题划分为多个子问题,该子问题基于从AP路径遇到陷阱问题导出的SRLG冲突链路集。我们的算法利用现有的AP搜索结果和并行执行来实现更快的路径查找。我们在一个多核CPU平台上使用生成的拓扑进行了广泛的模拟。仿真结果表明,在搜索速度较高的情况下,该算法的性能优于其它算法。
\section{未来研究工作与展望}

结论应是作者在学位论文研究过程中所取得的创新性成果的概要总结,不能与摘要混为一谈。
学位论文结论应包括论文的主要结果、创新点、展望三部分,在结论中应概括论文的核心观点,
明确、客观地指出本研究内容的创新性成果(含新见解、新观点、方法创新、技术创新、理论创新),
并指出今后进一步在本研究方向进行研究工作的展望与设想。
对所取得的创新性成果应注意从定性和定量两方面给出科学、准确的评价,分(1)、(2)、(3)…条列出,宜用“提出了”、“建立了”等词叙述。