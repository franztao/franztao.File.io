\section{选题背景和研究意义}
\subsection{选题研究背景}



\begin{frame}{目录}
    \setbeamertemplate{section in toc}[sections numbered]
    \tableofcontents[currentsection,hideothersubsections]
\end{frame}
\addtocounter{framenumber}{-1}  %目录页不计算页码

\begin{frame}
\frametitle{研究背景}
\begin{itemize}
  \item 在SDN控制器中,往往需要考虑在多约束下的路由问题。 特别在业务发放的场景中,为了到达抗故障的效果,需要为业务寻找工作与保护两条路径。
  \item 现实中,由于设备自身或者环境等因素,有时会引发物理网络故障,进而影响NFV租户虚拟网络的服务。因此需要为向租户虚拟网络嵌入请求提供一个高可生存性的保护。
\end{itemize}
\end{frame}

\subsection{研究意义}
\begin{frame}
\frametitle{研究意义}
\begin{itemize}
  \item 为了保证网络的通信质量,通常需要考虑时延,跳数,以及保证工作与备份路径间的不相交约束,多约束下的路由问题对实现网络资源实现高效快速调配和利用是具有重要的意义。
  \item 研究单物理节点故障可生存性虚拟网络嵌入问题,有利于提高NFV网络的可靠性,容错性,鲁棒性和可生存性,当物理层的网络节点出现故障时,能快速完全的保证原网络服务需求正常运行,这对网络的可生存性研究领域有重要的研究含义。
\end{itemize}
\end{frame}

%\begin{frame}
%\frametitle{研究现状}
%\begin{itemize}
%  \item 不相交路径问题
%  \item 可生存性虚拟网络嵌入问题
%\end{itemize}
%\end{frame}

%\begin{frame}
%\frametitle{研究工作}
%\begin{itemize}
%  \item 不相交路径问题,创新提出,SRLG冲突链路集合,分而治之
%  \item 可生存性虚拟网络嵌入问题,星型图,动态规划节点嵌入
%\end{itemize}
%\end{frame}
