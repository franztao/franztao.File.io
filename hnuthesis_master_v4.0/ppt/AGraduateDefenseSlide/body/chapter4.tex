\section{总结}
%\subsection{本文工作总结}
\begin{frame}{目录}
    \setbeamertemplate{section in toc}[sections numbered]
    \tableofcontents[currentsection,hideothersubsections]
\end{frame}
\addtocounter{framenumber}{-1}  %目录页不计算页码
  
\begin{frame}
\frametitle{总结}
\begin{enumerate}
  \item 首先,为了提高主路径传输的可生存性,全面研究了现在网络故障存在的各种故障类型的特点,包括主动式和被动式。针对这些故障恢复机制方式,设计了快速的分而治之的快速SRLG不相交路径对算法。
  \item 本文提出了一种拓扑图在存在陷阱问题情况下求解Min-Min SRLG不相交路由问题的高效算法,为了降低搜索的复杂性,我们创新性提出了一种分而治之的解决方案,将原Min-Min SRLG 不相交路由问题划分为多个子问题,该子问题基于从AP路径上遇到陷阱问题时导出的SRLG冲突链路集。我提出的算法利用现有的AP搜索结果SRLG冲突链路集,来并行执行实现更快的路径查找。
  \item 并且本文在一个多核CPU平台上使用合成的拓扑进行了真实的仿真模拟。仿真结果表明,在搜索速度比较下,该算法的查找性能优于其它现有算法。
  \item 其次,我们分析虚拟网络嵌入问题的本质特征,从而提出适合星型分解动态规划节点嵌入的可生存性虚拟网络嵌入算法,以解决原先可生存性虚拟网络嵌入算法无法解决的时间复杂性和低资源利用率等问题。
  \item 本文在可生存性虚拟网络嵌入问题中,引入网络节点带有特定功能类型的限制条件,创新性提出一种星型分解动态规划节点嵌入的启发式算法,在虚拟和物理星型图之间的权重设置上,考虑网络资源的利用率和网络节点开启代价。
  \item   提出的算法能快速的实现虚拟网络嵌入的可生存性需求,仿真结果表明,我提出的算法与其他现有算法效果比,虚拟嵌入可生存性请求的成功率更高,嵌入的物理资源消耗更低,物理资源的利用率更高。
\end{enumerate}
\end{frame}
