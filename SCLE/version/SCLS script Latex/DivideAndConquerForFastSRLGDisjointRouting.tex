%Regular papers (max. 12 page submission; 12 pages in proceedings): full paper describing a research contribution to dependable computing, including experimental work focused on implementation and evaluation of existing dependability techniques.
%The number of pages indicated above includes everything: title page, text, references, figures, appendices, etc.
%The first page must include the title of the paper, the type of the paper (Regular / Practical Experience / Tool), and a maximum 150-word abstract.


%https://dsn2018.uni.lu/main-track-call-papers/

%Pages should be numbered.

\documentclass[10pt,conference]{IEEEtran}
%\IEEEoverridecommandlockouts
% The preceding line is only needed to identify funding in the first footnote. If that is unneeded, please comment it out.
%\usepackage{latex8}
%\usepackage{times}
\usepackage{amssymb}
\usepackage{amsmath}
\usepackage{graphicx}
\usepackage{subfigure}
\usepackage{epstopdf}
\usepackage{subfigure}
%\usepackage[nocompress]{cite}
\usepackage{algorithm}
\usepackage{algorithmic}
%\usepackage{titlesec}
\usepackage{multirow}
\usepackage{tabularx}
\usepackage{booktabs}
\usepackage{threeparttable}
\usepackage[normalem]{ulem}
\usepackage{paralist}
\usepackage{url}
%\usepackage{CJK}
\usepackage{amsthm}
%\pagestyle{plain}
\usepackage{geometry}
\usepackage{tikz}
\usepackage{setspace}
%\usepackage[pdftex,active,tightpage]{preview}
%\setlength\PreviewBorder{0mm}
\usepackage{booktabs} % For formal tables


%\geometry{left=1.40cm,right=1.40cm,top=1.33cm,bottom=2.1cm}
\linespread{0.986}
\setlength{\abovedisplayskip}{0.5 mm}
\setlength{\belowdisplayskip}{0.5 mm}
\setlength{\abovecaptionskip}{0.0pt}
\setlength{\belowcaptionskip}{0.0pt}
\geometry{left=1.5cm,right=1.5cm,top=1.5cm,bottom=1.55cm}


%%\geometry{left=1.40cm,right=1.40cm,top=1.33cm,bottom=2.1cm}
%%\linespread{0.986}
%\setlength{\abovedisplayskip}{0.1 mm}
%\setlength{\belowdisplayskip}{0.1 mm}
%%\setlength{\abovecaptionskip}{0.0pt}
%%\setlength{\belowcaptionskip}{0.0pt}
%%\titlespacing{\section}{0pt}{4pt}{0 pt}
%%\titlespacing{\subsection}{0pt}{2pt}{0pt}
%%\titlespacing{\subsubsection}{0pt}{2pt}{0pt}
\usepackage[colorlinks,linkcolor=red,anchorcolor=blue]{hyperref}


\newcommand{\songti}{\CJKfamily{song}}
\newcommand{\rev}[1]{\uwave{#1}}
%\newcommand{\rev}[1]{#1}
%\renewcommand{\algorithmicrequire}{\textbf{Input:}}
%\renewcommand{\algorithmicensure}{\textbf{Output:}}
\newcommand{\del}[1]{\sout{#1}}  %revise the text
%\newcommand{\del}[1]{}
\newcommand{\note}[1]{{\sffamily\itshape\bfseries\uline{#1}}}
%\newcommand{\note}[1]{}
%\newcommand{\songti}{\CJKfamily{song}}
%\newcommand{\rev}[1]{\textcolor{blue}{{#1}}}%revise the text with blue color, you can use this one.
\newcommand{\revtao}[1]{\textcolor{orange}{{#1}}}%revise the text with blue color, you can use this one.
%\newcommand{\revtao}[1]{\textcolor{black}{{#1}}}%revise the text with blue color, you can use this one.
%\newcommand{\del}[1]{\sout{#1}}  %revise the text
%\newcommand{\del}[1]{}
%\newcommand{\note}[1]{{\sffamily\itshape\bfseries\uline{#1}}}



\newcommand{\CSLI}{SRLG Conflicting Link Set}
\newcommand{\CI}{SCLS}
\newcommand{\TopoNum}{7}
\newtheorem{theorem}{Theorem}
\newtheorem{lemma}{Lemma}
\newtheorem{definition}{Definition}



\begin{document}
%\begin{CJK*}{GBK}{song}

\title{Divide And Conquer For Fast SRLG Disjoint Routing}
\thanks{The work is supported  by the  National Natural Science Foundation of China under Grant Nos.61572184, 61725206, 61472130, 61472131, and 61772191,  Hunan Provincial Natural Science Foundation of China under Grant No.2017JJ1010,  Science and Technology Key Projects of Hunan Province under Grant No.2015TP1004 and No.2016JC2012,  U.S. ONR N00014-17-1-2730, NSF ECCS 1408247, CNS 1526843, and ECCS 1731238,  the open project funding (CASNDST201704) of CAS Key Lab of Network Data Science and Technology, Institute of Computing Technology, Chinese Academy of Sciences, and the foundation of Key Laboratory of Machine Intelligence and Advanced Computing of the Ministry of Education under Grant No.MSC-201708A.}

{
\author{\IEEEauthorblockN{Kun Xie}
\IEEEauthorblockA{College of Computer Science and\\ Electronic Engineering \\
Hunan University\\
Chang Sha, China \\
xiekun@hnu.edu.cn}
\and
\IEEEauthorblockN{Heng Tao}
\IEEEauthorblockA{College of Computer Science and\\ Electronic Engineering \\
Hunan University\\
Chang Sha, China\\
franztaoheng@gmail.com}
\and
\IEEEauthorblockN{Xin Wang}
\IEEEauthorblockA{Department of Electrical and\\ Computer Engineering \\
Stony Brook University\\
New York, USA \\
x.wang@stonybrook.edu}
\and
\IEEEauthorblockN{Gaogang Xie}
\IEEEauthorblockA{Institute of Computing Technology \\
Chinese Academy of Sciences\\
Bei Jing, China \\
xie@ict.ac.cn}
\and
\IEEEauthorblockN{Jigang Wen}
\IEEEauthorblockA{Institute of Computing Technology\\
Chinese Academy of Sciences\\
Bei Jing, China \\
wenjigang@ict.ac.cn}
\and
\IEEEauthorblockN{Jiannong Cao}
\IEEEauthorblockA{Department of computing \\
The Hong Kong Polytechnic University\\
Hong Kong, China \\
csjcao@comp.polyu.edu.hk}
\and
\IEEEauthorblockN{Zheng Qin}
\IEEEauthorblockA{College of Computer Science and\\ Electronic Engineering \\
Hunan University\\
Chang Sha, China \\
zqin@hnu.edu.cn}
}

\maketitle
%\vspace{-3em}
%\vspace{-0.6in}
\begin{abstract}
Network virtualization has been recognized as a promising solution to enable the rapid deployment of customized services by building multiple Virtual Networks (VNs) on a shared substrate network. Whereas various VN embedding schemes have been proposed to allocate substrate resources to VN requests, little work has been done to provide backup mechanisms in case of substrate network failures. In a virtualized infrastructure where physical resources are shared, a single physical server failure will terminate several virtual serves and crippling the virtual infrastructures which contained those virtual servers. To guarantee some level of survivability, each virtual infrastructure, at instantiation, should be augmented with backup virtual nodes and links. Provisioning survivability to requested embeded virtual network(VN) which is embedded in substrate networks (SN) in a resource efficient way is important. we investigate the problem of backup network provision for embedded VN embedding and propose one star graph decomposition and dynamic programming for node mapping, in order to provide the same degree of protection against a single node failure with less substrate resources. Simulation experiments show that we proposed scheme make better high acceptance ratio and utilization of substrate resources than other backup schemes.
\end{abstract}

\begin{IEEEkeywords}
Shared Risk Link Groups (SRLG), SRLG-Disjoint Routing, Max-Flow Min-Cut Theorem
\end{IEEEkeywords}


\input introduction
\input related
\input preniminaries
\input problem
\input overview
\input conflict
%\input  srng
\input complete
\input performance
\input conclusion
%\end{spacing}p
\section*{Acknowledgment}
%\end{spacing}p
The work is supported  by the  National Natural Science Foundation of China under Grant Nos.61572184, 61725206, 61472130, 61472131, and 61772191,  Hunan Provincial Natural Science Foundation of China under Grant No.2017JJ1010,  Science and Technology Key Projects of Hunan Province under Grant No.2015TP1004 and No.2016JC2012,  U.S. ONR N00014-17-1-2730, NSF ECCS 1408247, CNS 1526843, and ECCS 1731238.


%\input addition
\bibliographystyle{ieeetr}
\bibliography{Bib}

\end{document}
%\section{EeVSN Formulation}
After designing an EVN, in this section, we also formulate
its embedding problem as an MILP fitting for both FD-EVN
and FI-EVN embedding, since the difference between them in
embedding approach could be reconciled by a general resources
sharing constraint. More details would be elaborated in the following
part.

with respect to the node embedding, we
have the assumption that all the virtual nodes in EVN should be
mapped on physically isolated substrate nodes.

different node failure scenarios.
Thus, rather than solving the link embedding problem based
on the multi-commodity flow LP, we need consider the resource
sharing issues, which is computational intractable. As a result,
heuristic algorithms with acceptable computation complexity.


%\input{EeVSNDesign}
