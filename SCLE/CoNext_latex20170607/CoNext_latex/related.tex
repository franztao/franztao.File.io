\section{Related work}
\label{sec:Related}
 A SRLG is a group of network links that share a common physical resource (cable, conduit, node or substructure) whose failure will cause the failure of all links of the group. For path protection, although some link/node-disjoint routing algorithms have been proposed so far \cite{suurballe1984quick,bhandari1997optimal,li1990complexity,guo2003link,xu2004finding,beshir2011variants,guo2013finding,hu2003diverse}, there are very limited studies on SRLG-disjoint routing. When the SRLG contains only one link, the SRLG-disjoint routing problem can be reduced to a link-disjoint routing problem, while node-disjoint routing problem could be transformed into a link-disjoint routing problem through node splitting \cite{ford2015flows}.
 % and} is also a simple and specific SRLG-disjoint routing problem.
Since a SRLG group generally includes more than one link and a network link may belong to several SRLGs, finding a pair of SRLG-disjoint paths is much more difficult than finding a pair of link/node disjoint paths.


To solve a SRLG-disjoint routing problem, one possible way is to form an Integer Linear Programming (ILP)\cite{hu2003diverse} problem to optimally select both AP and BP paths through the branch-and-bound search. This method incurs a high time complexity, and is not feasible for large networks. To reduce the complexity, APF-based heuristics are shown to be able to achieve near-optimal solutions to the Min-Min SRLG-disjoint problem \cite{oki2002disjoint,li2002fiber,eppstein1998finding}. In these APF-based heuristics, an AP is found first by using the Dijkstra algorithm (or any other shortest path algorithms) without considering the need to find a corresponding BP, and the BP is found (again using the Dijkstra algorithm for example) after removing the links or nodes along the AP or share the risk with the AP.

However, as a major challenge in using the APF heuristic, once an AP is found, a SRLG disjoint BP  may not be able to be found even though a pair of disjoint paths do exist in the network. This is called the "trap" problem, which can happen even if the network is highly connected \cite{laborczi2001solving}, and certainly cannot be ignored in a sparsely-connected network. There are two kinds of trap, unavoidable and avoidable trap. Unavoidable traps are constraints forced by the topology, and cannot be solved by any algorithm. If a network is not 2-edge connected, then there is no algorithm that can guarantee the presence of two SRLG disjoint pathes in the topology. On the other hand, an avoidable trap occurs when an SRLG-disjoint path pair exists between two nodes but  cannot be found due to the shortcomings of the routing algorithms. In this paper, we consider the avoidable trap.

As an extension to a simple-APF, the KSP algorithm\cite{eppstein1998finding}  is proposed to deal with the trap problem for node/link disjoint path. Although it is one of the most effective algorithms to deal with the trap problem,  its performance suffers in a large network. After the current candidate AP encounters the trap problem, the next candidate AP to be tested is selected solely based on the path length, without considering which link (or links) along the current candidate AP has caused the failure in finding disjoint BP. As a result, a large number of paths often need to be tested in order to find a disjoint path pair, which introduces a large time complexity. Different from KSP, for an  AP encountering the trap problem, we apply the SRLG Conflicting Link Set derived from this AP path to guide the future AP testing. This helps to largely reduce the time complexity in finding the alternative paths.

Other SRLG-disjoint routing algorithms, including \cite{xu2003trap,rostami2012msdp,rostami2007cose,datta2008graph,xu2003new,todimala2004imsh}, search for the maximal SRLG disjoint routing paths based on ant colony optimization\cite{rostami2012msdp}. As AP and BP may share same risks, the solutions found through this type of approaches are not reliable. Our algorithm targets to find complete SRLG disjoint paths. The work in~\cite{datta2008graph} transforms the SRLG disjoint routing problem to the link-disjoint routing problem and then exploits the link-disjoint routing algorithm to solve the problem. However, only certain SRLG styles (e.g, star-style) can be transformed to the link-disjoint, which limits the application of this scheme. Rather than calculating the conflicting link set, when AP encounters a trap problem, CoSE\cite{rostami2007cose} tries to find a SRLG set that no AP going through the SRLG set can find the SRLG BP path. CoSE first looks for the SRLG set commonly shared by APs through multiple rounds of search, then  partitions the original problem based on the SRLG set to search for a SRLG-disjoint path pair.  Without utilizing the risk sharing feature in SRLG, this exhaustive search in CoSE incurs very high computational overhead.



Besides the long computation time, most of current SRLG-routing algorithms consider a single scenario that one link belongs to only one SRLG. Different from existing studies\cite{rostami2007cose,datta2008graph}, this paper tries to guide the search of alternative AP paths by analyzing the links on the path of the AP caught into the "trap". More specifically, this paper proposes a divide and conquer algorithm to partition the original problem into multiple sub-problems which can be executed in parallel to largely speed up the searching process. Moreover, our SRLG-routing algorithm does not have any constraint on the SRLG-style and can efficiently handle complex SRLG situation that a link belongs to multiple SRLGs.
