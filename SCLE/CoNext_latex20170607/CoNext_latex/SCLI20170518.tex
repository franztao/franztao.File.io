\documentclass[10pt,sigconf]{acmart}
%\usepackage{times}
\usepackage{amssymb}
\usepackage{amsmath}
\usepackage{graphicx}
\usepackage{subfigure}
\usepackage{epstopdf}
\usepackage{subfigure}
%\usepackage[nocompress]{cite}
\usepackage{algorithm}
\usepackage{algorithmic}
%\usepackage{titlesec}
\usepackage{multirow}
\usepackage{tabularx}
\usepackage{booktabs}
\usepackage{threeparttable}
\usepackage[normalem]{ulem}
\usepackage{paralist}
\usepackage{url}
%\usepackage{CJK}
\usepackage{amsthm}
%\pagestyle{plain}
%\usepackage{geometry}
\usepackage{tikz}
\usepackage{setspace}
%\usepackage[pdftex,active,tightpage]{preview}
%\setlength\PreviewBorder{0mm}

%%\geometry{left=1.40cm,right=1.40cm,top=1.33cm,bottom=2.1cm}
\linespread{0.986}
\setlength{\abovedisplayskip}{0.5 mm}
\setlength{\belowdisplayskip}{0.5 mm}
\setlength{\abovecaptionskip}{0.0pt}
\setlength{\belowcaptionskip}{0.0pt}
%\titlespacing{\section}{0pt}{4pt}{0 pt}
%\titlespacing{\subsection}{0pt}{2pt}{0pt}
%\titlespacing{\subsubsection}{0pt}{2pt}{0pt}
%\usepackage[colorlinks,linkcolor=red,anchorcolor=blue]{hyperref}
\hypersetup{colorlinks,linkcolor=red,anchorcolor=blue}


\newcommand{\songti}{\CJKfamily{song}}

\newcommand{\rev}[1]{\textcolor{blue}{{#1}}}%revise the text with blue color, you can use this one.

\newcommand{\revtao}[1]{\textcolor{orange}{{#1}}}%revise the text with blue color, you can use this one.

\newcommand{\del}[1]{\sout{#1}}  %revise the text
%\newcommand{\del}[1]{}

\newcommand{\note}[1]{{\sffamily\itshape\bfseries\uline{#1}}}

\newcommand{\CSLI}{SRLG Conflicting Link Set}
\newcommand{\CI}{SCLS}


% Copyright
%\setcopyright{none}
\setcopyright{acmcopyright}
%\setcopyright{acmlicensed}
%\setcopyright{rightsretained}
%\setcopyright{usgov}
%\setcopyright{usgovmixed}
%\setcopyright{cagov}
%\setcopyright{cagovmixed}


% DOI
%\acmDOI{10.475/123_4}

% ISBN
%\acmISBN{123-4567-24-567/08/06}

%Conference
\acmConference[CoNEXT'17]{The 13th International Conference on emerging Networking EXperiments and Technologies}{Dec.2017}{Seoul/Incheon, South Korea}
\acmYear{2017}
\copyrightyear{2017}

%\acmPrice{15.00}


\begin{document}
\title{Divide And Conquer For Fast SRLG Disjoint Routing}
%\titlenote{Produces the permission block, and copyright information}
%\subtitle{Extended Abstract}

\author{Kun Xie}
%\authornote{Note}
%\orcid{1234-5678-9012}
\affiliation{%
  \institution{Department of Electrical and Computer Engineering, Stony Brook University, New York}}
\email{kun.xie@stonybrook.edu}

\author{Heng Tao}
%\orcid{1234-5678-9012}
\affiliation{%
  \institution{Department of Electrical and Computer Engineering, Stony Brook University, New York}
}
\email{franztaoheng@gmail.com}

\author{Xin Wang}
%\orcid{1234-5678-9012}
\affiliation{%
  \institution{Department of Electrical and Computer Engineering, Stony Brook University, New York}
}
\email{x.wang@stonybrook.edu}

\author{Gaogang Xie}
%\orcid{1234-5678-9012}
\affiliation{%
  \institution{Institute of Computing Technology, Chinese Academy of Sciences, Bei Jing}
}
\email{xie@ict.ac.cn}
\author{Jiannong Cao}
%\orcid{1234-5678-9012}
\affiliation{%
  \institution{Department of computing, The Hong Kong Polytechnic University, Hong Kong}
}
\email{csjcao@comp.polyu.edu.hk}

% The default list of authors is too long for headers}
%\renewcommand{\shortauthors}{F. Lastname et al.}


\begin{abstract}
Ensuring transmission survivability is a crucial problem for high-speed networks. Path protection is a fast and capacity-efficient approach for increasing the availability of end-to-end connections. The emerging SDN infrastructure makes it feasible to provide diversity routing in a practical network. For more robust path protection, it is desirable to provide an alternative path that does not share any risk resource with the active path. We consider finding the SRLG-Disjoint paths, where a Shared Risk Link Group (SRLG) is a group of network links that share a common physical resource whose failure will cause the failure of all links of the group. Since the traffic is carried on the active path most of time, it is useful that the weight of the shorter path of the disjoint path pair is minimized, and we call it  Min-Min SRLG-Disjoint problem. We prove this problem is NP-complete. The key issue faced by SRLG-Disjoint routing is the trap problem, where the SRLG-disjoint backup path (BP) can not be found after an active path (AP) is decided. Based on the min-cut of the graph, we  design an  efficient algorithm that can  take advantage of existing search results to quickly look for the SRLG-Disjoint path pair. Our performance studies demonstrate that our algorithm can outperform other approaches with a higher routing performance while also at a much faster speed.
\end{abstract}

%
% The code below should be generated by the tool at
% http://dl.acm.org/ccs.cfm
% Please copy and paste the code instead of the example below.
%%
%\begin{CCSXML}
%<ccs2012>
% <concept>
%  <concept_id>10010520.10010553.10010562</concept_id>
%  <concept_desc>Computer systems organization~Embedded systems</concept_desc>
%  <concept_significance>500</concept_significance>
% </concept>
% <concept>
%  <concept_id>10010520.10010575.10010755</concept_id>
%  <concept_desc>Computer systems organization~Redundancy</concept_desc>
%  <concept_significance>300</concept_significance>
% </concept>
% <concept>
%  <concept_id>10010520.10010553.10010554</concept_id>
%  <concept_desc>Computer systems organization~Robotics</concept_desc>
%  <concept_significance>100</concept_significance>
% </concept>
% <concept>
%  <concept_id>10003033.10003083.10003095</concept_id>
%  <concept_desc>Networks~Network reliability</concept_desc>
%  <concept_significance>100</concept_significance>
% </concept>
%</ccs2012>
%\end{CCSXML}
%\ccsdesc[500]{Computer systems organization~Embedded systems}
%\ccsdesc[300]{Computer systems organization~Redundancy}
%\ccsdesc{Computer systems organization~Robotics}
%\ccsdesc[100]{Networks~Network reliability}

% We no longer use \terms command
%\terms{Theory}

\keywords{Shared Risk Link Groups (SRLG); SRLG-Disjoint Routing; Max-Flow Min-Cut Theorem}

\maketitle
%\begin{spacing}{0.95}%%行间距变为single-space
\input introduction
\input related
\input preniminaries
\input problem
\input overview
\input conflict
\input complete
\input performance
\input conclusion
%\end{spacing}


%\input addition
\bibliographystyle{ACM-Reference-Format}
\bibliography{Bib}

\end{document}
