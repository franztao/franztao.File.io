\section{EeVSN Design Formulation}
The software-defined NFV architecture further offers agile traffic steering and joint optimization of network functions and resources In this section, we define the EeVSN design problem as follows: for a given VN request with N task nodes, the VN is embeded in substrate network SN and running service $s_i(s_i\in S)$, enhance the VN with one additional node and a set of appropriate links to connect these nodes, and reserve sufficient computing and communication resources in these nodes and links to guarantee the restorability of VN request after a facility node failure.

write the bandwith and computting resource between FI and FD.

\subsection{VN Reqeust Model}
A task graph for a VN request is an undirected attributed
graph, denoted asGV (V V ,EV ), where V V corresponds to a set
of task nodes andEV denotes a set of bidirectional edges among
the VN nodes. Each task node specifies the needed computing
resource cVs
, and each edge specifies the requested amount of
communication (bandwidth) resource bV
st .

\subsection{EeVSN Design Formulation}
With the discussion above, for a given N-nodes VN, EVN is designed within an Edit Grid with N+1 nodes through properly
selecting the necessary links between these nodes and dimensioning the resources requirements associated with these nodes
and links. Our design objective is to minimize the total amount of such resources while still guaranteeing that if a node fails
(that is, the node and its adjacent links are removed in the Edit Grid), we can still assign each node/link in VN to a node/link in
the EVN, that has sufficient CP.CM. resources respectively.

Furthermore, there are two different cases. If re-embedding or
migrating the unaffected task node is allowed for failure recovering,
it is known as the FD-EVN design problem.Otherwise, after
each failure, if the failed node is restored in the only one backup
node without migrating other unaffected node, it is referred to
as the FI-EVN design problem. Generally speaking, designing
FD-EVN is a combinatorial optimization problem and needs to
be investigated in depth, while FI-EVN exists exclusively and
could be figured out easily (meaning not very clear).

In conclusion, the formulation indicates that this problem is
a BQP problem and the solution of this problem is a series of
permutation matrix  which determine the
migration reassigning approach after each node failure. Thus, its
computation complexity is (N)N , whereN is the computation
complexity for one node failure.
